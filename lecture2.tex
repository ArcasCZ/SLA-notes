\section{Druhá přednáška}

\subsection{Algebraické struktury}
Algebraická struktura je množina, na které máme jednu, nebo více operací
a tyto operace mají nějaké vlastnosti. Obecně $(G, *)$ je algebraická
struktura na množině G s operací *. Algebraických struktur je mnoho, nas bude
zajímat převážně Grupa a Pole. Pokud bychom z následujicí definice grupy vypustili
všechny 3 podmínky, jednalo by se o tzv. Grupoid (také označován jako Magma). Při splnění první
podmínky tedy Magma a 1. podmínka, dostáváme tzv Pologrupu. Následně Pologrupou a
splněním podmínky číslo 2 dostáváme Monoid.

\subsubsection{Grupa}
\begin{definition}[Grupa]
    Grupa $(G, *)$ je algebraická struktura s jednou operací $*: G \times G \rightarrow G$,
    kde operace $*$ splňuje následujicíc vlastnosti:
    \begin{enumerate}
        \item $a * (b * c) = (a * b) * c \; \forall a, b, c \in G$ \hfill Asociativita
        \item $\exists e \in G: e * a = a * e = a \; \forall a \in G$ \hfill Neutrální prvek
        \item $\forall a \in G, \exists a^{-1} \in G: a * a^{-1} = a^{-1} * a = e$ \hfill
        Inverzní prvky
    \end{enumerate}
\end{definition}

\begin{definition}[Komutativní \uv{Abelova} grupa]
    Pokud k požadovaným vlastnostem operace $*$ tvořící grupu přidáme ještě
    čtvrtou vlastnost:
    \begin{enumerate}[start=4]
        \item $a * b = b * a \; \forall a, b \in G$ \hfill Komutativita
    \end{enumerate}
    Dostaneme tzv. Abelovskou grupu.
\end{definition}

Jako příklady grupy můžeme uvést $(\mathbb{Z, +})$, $(\mathbb{Q} \smallsetminus \{0\}, \cdot)$
$(\mathbb{Q}, +)$, $(\mathbb{R}, +)$, $(\mathbb{R} \smallsetminus \{0\}, \cdot)$
všechny tyto příklady jsou dokonce abelovskou grupou. Zajímavé je zamyslet se nad příkladem
neabelovské grupy, kterým může být například grupa permutací (permutace s operací skládání s třemi
a více prvky). Dalším příkladem neabelovské grupy je množina čtvercových regulárních matic s operací
násobení.

\begin{theorem}
    Neutrální prvek je jediný.
\end{theorem}
\begin{proof}
    Předpokládejme, že $e_1$ a $e_2$ jsou neutrální prvky. Budeme li chtít na tyto dva neutrální
    prvky aplikovat operaci $*$ podle definice neutrálního prvku vezmeme $e_1$ jako neutrální a
    dostáváme:
    $$e_1 * e_2 = e_2$$
    Zároveň ale můžeme podle definice neutrálního prvku vzít $e_2$ jako neutrální a v tom případě
    dostáváme:
    $$e_1 * e_2 = e_1$$
    Z toho vyplývá, že $e_1$ a $e_2$ jsou tentýž prvek a nemůže tedy nikdy exisovat více než
    jeden neutrální prvek.
\end{proof}

\subsubsection{Pole}
\begin{definition}[Pole]
    Pole $(F, +, \cdot)$ je algebraická struktura se dvěma operacemi, kde množina $F$ má alespoň
    dva prvky, operace $+$ splňuje následujicí vlastnosti\footnote{Všiměte si, že jsou velmi podobné
    požadovaným vlastnostem na operaci $*$ z definice Abelovy grupy.}:
    \begin{enumerate}
        \item $a + (b + c) = (a + b) + c \; \forall a, b, c \in F$ \hfill Asociativita
        \item $\exists 0_f \in F: 0_f + a = a + 0_f = a \; \forall a \in F$ \hfill Neutrální prvek
        \item $\forall a \in F, \exists -a \in F: a + (-a) = -a + a = 0_f$ \hfill Inverzní prvky
        \item $a + b = b + a \; \forall a, b \in F$ \hfill Komutativita
    \end{enumerate}
    a zároveň operace $\cdot$ splňuje:
    \begin{enumerate}
        \item $a \cdot (b \cdot c) = (a \cdot b) \cdot c \; \forall a, b, c \in F$ \hfill Asociativita
        \item $\exists 1_f \in F: 1_f \cdot a = a \cdot 1_f = a \; \forall a \in F$ \hfill Neutrální prvek
        \item $\forall a \in F \smallsetminus \{0_f\}, \exists a^{-1} \in F: a \cdot
        a^{-1} = a^{-1} \cdot a = 1_f$ \hfill Inverzní prvky
        \item $a \cdot (b + c) = a \cdot b + a \cdot c \wedge (b + c) \cdot a = b \cdot a + c \cdot a
        \; \forall a, b,c \in F$ \hfill Distributivita

    \end{enumerate}
    \label{def:field}
\end{definition}

\begin{definition}[Komutativní pole]
    Pokud se jedná o pole a navíc je operace $\cdot$ komutativní, jedná se o komutativní pole:
    \begin{enumerate}[start=5]
        \item $a \cdot b = b \cdot a \; \forall a, b \in F$ \hfill Komutativita
    \end{enumerate}
\end{definition}

Zatím jediným příkladem pole, který z přednášek známe je $(\mathbb{Q}, +, \cdot)$.\footnote{Dalším
příkladem by mohlo být pole racionálních funkcí $\mathbb{Z}(X)$, které bylo později velmi okrajově
zmíněno na přednášce.}

\begin{definition}[Uspořádané pole]
    Řekneme, že pole $F$ je uspořádané, jestliže v něm existuje $P \subseteq F$ tak, že
    je li $x, y \in P$ platí $x + y \in P \wedge x \cdot y \in P$ a dále $\forall x \in F$
    platí, že splňuje právě jednu z následujicích podmínek:
    \begin{enumerate}
        \item $x \in P$
        \item $-x \in P$
        \item $x = 0_F$
    \end{enumerate}
    \label{def:ordered_field}
\end{definition}

Jinak řečeno, uspořádané pole bude takové, ve kterém je možné nějakým způsobem vybrat \uv{kladnou}
pod\-mno\-ži\-nu. Příklady uspořádaných polí: $\mathbb{Q}, \mathbb{R}, \mathbb{Z}(x)$

Mejme uspořádané pole dle definice \ref{def:ordered_field}, potom
zavedeme relaci $<$ následovně:
$$a < b, \text{jestliže}\,b - a \in P$$

Taková relace je ostré uspořádání\footnote{To znamená, že je tato relace ireflexivní
a tranzitivní}.

\begin{definition}[Husté pole]
    Řekneme, že pole F je husté, jestliže $\forall a, b \in F, a < b$ existuje $c \in F$
    takové, že $a < c < b$.
\end{definition}

Příklady hustého pole: $\mathbb{Q}, \mathbb{R}$.

\begin{definition}[Archimédovské pole]
    Řekneme, že uspořádané pole $F$ je archimédovské, jestliže:
    $$\forall x, y \in P\,\exists n\in\mathbb{N}, n \cdot x > y $$
\end{definition}

Příklady archimédovských polí: $\mathbb{Q}, \mathbb{R}$

\subsubsection{Konečná pole}
\begin{definition}
    Konečné pole je pole $(F, +, \cdot)$, kde množina $F$ má konečný počet prvků.
\end{definition}
\begin{theorem}[Existence konečného pole]
    \label{the:field}
    Konečné pole $(F, +, \cdot)$ existuje právě tehdy, když $|F| = p^k$, kde $p$ je
    prvočíslo a $k \in \mathbb{N}$. Toto konečné pole je zároveň jediné.
\end{theorem}

Z věty \ref{the:field} vyplývá, že existují konečná pole se dvěma prvky, třemi prvky, se
čtyřmi prvky, s pěti prvky, ale ne se šesti, protože 6 není ani prvočíslo, ani mocnina prvočísla.

Konečná pole budeme značit zdvojeným fontem a počtem prvků v dolním indexu např. $\mathbb{F}_{11}$

Konečná pole si rozdělíme na dva případy a to na prvočíselná pole a na neprvočíselná pole.

Abychom porozuměli konečným polím a mohli s nimi pracovat, potřebujeme vědět,
jak na nich fungují operace $+$ a $\cdot$.

\subsubsection*{Prvočíselná pole}
\begin{definition}[Prvočíselné pole]
    Prvočíselné pole je konečné pole $(F, +, \cdot)$, kde $|F| = p$, p je prvočíslo. Tedy
    všechny případy, kdy pro $k$ z věty \ref{the:field} platí že $k=1$.
\end{definition}

Například v prvočíselném poli $\mathbb{F}_{2}$
máme 2 prvky a tyto prvky můžeme označit jak chceme, pro praktické počítání je však nejlepší
označit tyto prvky čísly, v tomto případě od $0$ do $1$, kde $0$ bude hrát roli hodnoty nula a
$1$ roli hodnoty jedna, tak jak potřebujeme.

\begin{example}[Prvočíselné pole $\mathbb{F}_{2}$]
    $$\mathbb{F}_2 = \{0, 1\}$$
    \begin{table}[h]
        \centering
        \begin{tabular}{|c|c|l|l|l|c|c|l|}
        \hline
        $+$ & $0$ & $1$ &  &  & $\cdot$ & $0$ & $1$ \\ \hline
        $0$ & $0$ & $1$ &  &  & $0$     & $0$ & $0$ \\ \hline
        $1$ & $1$ & $0$ &  &  & $1$     & $0$ & $1$ \\ \hline
        \end{tabular}
        \caption{Operace $+$ a $\cdot$ nad $\mathbb{F}_{2}$}
        \label{tab:F2}
    \end{table}

    Můžeme si všimnout, že u obou operací v tomto případě vlastně počítáme modulo 2,
    tedy modulo počet prvků pole, tato vlastnost platí obecně u prvočíselných polí.
\end{example}


\begin{example}[Prvočíselné pole $\mathbb{F}_{5}$]
    $$\mathbb{F}_5 = \{0, 1, 2, 3, 4\}$$
\begin{table}[h]
    \centering
    \begin{tabular}{|c|c|c|c|c|c|c|c|c|c|c|c|c|c|}
    \hline
    $+$ &
      $0$ &
      $1$ &
      $2$ &
      $3$ &
      $4$ &
       &
       &
      $\cdot$ &
      $0$ &
      $1$ &
      $2$ &
      $3$ &
      $4$ \\ \hline
    $0$ &
      $0$ &
      $1$ &
      $2$ &
      $3$ &
      $4$ &
       &
       &
      $0$ &
      $0$ &
      $0$ &
      $0$ &
      $0$ &
      $0$ \\ \hline
    $1$ &
      $1$ &
      \cellcolor[HTML]{FFFFFF}$2$ &
      \cellcolor[HTML]{FFFFFF}$3$ &
      \cellcolor[HTML]{FFFFFF}$4$ &
      \cellcolor[HTML]{FFFFFF}$0$ &
       &
       &
      $1$ &
      $0$ &
      \cellcolor[HTML]{34FF34}$1$ &
      \cellcolor[HTML]{34FF34}$2$ &
      \cellcolor[HTML]{34FF34}$3$ &
      \cellcolor[HTML]{34FF34}$4$ \\ \hline
    $2$ &
      $2$ &
      \cellcolor[HTML]{FFFFFF}$3$ &
      \cellcolor[HTML]{FFFFFF}$4$ &
      \cellcolor[HTML]{FFFFFF}$0$ &
      \cellcolor[HTML]{FFFFFF}$1$ &
       &
       &
      $2$ &
      $0$ &
      \cellcolor[HTML]{34FF34}$2$ &
      \cellcolor[HTML]{34FF34}$4$ &
      \cellcolor[HTML]{34FF34}$1$ &
      \cellcolor[HTML]{34FF34}$3$ \\ \hline
    $3$ &
      $3$ &
      \cellcolor[HTML]{FFFFFF}$4$ &
      \cellcolor[HTML]{FFFFFF}$0$ &
      \cellcolor[HTML]{FFFFFF}$1$ &
      \cellcolor[HTML]{FFFFFF}$2$ &
       &
       &
      $3$ &
      $0$ &
      \cellcolor[HTML]{34FF34}$3$ &
      \cellcolor[HTML]{34FF34}$1$ &
      \cellcolor[HTML]{34FF34}$4$ &
      \cellcolor[HTML]{34FF34}$2$ \\ \hline
    $4$ &
      $4$ &
      \cellcolor[HTML]{FFFFFF}$0$ &
      \cellcolor[HTML]{FFFFFF}$1$ &
      \cellcolor[HTML]{FFFFFF}$2$ &
      \cellcolor[HTML]{FFFFFF}$3$ &
       &
       &
      $4$ &
      $0$ &
      \cellcolor[HTML]{34FF34}$4$ &
      \cellcolor[HTML]{34FF34}$3$ &
      \cellcolor[HTML]{34FF34}$2$ &
      \cellcolor[HTML]{34FF34}$1$ \\ \hline
    \end{tabular}
    \caption{Operace $+$ a $\cdot$ nad $\mathbb{F}_{5}$}
    \label{tab:F5}
    \end{table}

    Z definice pole\ref{def:field} vyplývá, že $(\mathbb{F}_5, +)$ musí tvořit Abelovskou grupu.
    V Abelovské grupě platí, že při rozepsání operace do tabulky je v každém sloupci
    a v každém řádku každý prvek obsažen právě jednou\footnote{V počátcích definic
    teorie grup se tato vlastnost používala pro definici grupy.}. Což si můžeme všimnout
    že zde platí.

    U operace $\cdot$ si můžeme všimnout, že bez prvního sloupce a bez prvního řádku
    (zeleně označená část) operace $\cdot$ tvoří grupu. Tato vlastnost u pole a jeho
    operace $\cdot$ platí vždy.
\end{example}

\subsubsection*{Neprvočíselná pole}
\begin{definition}[Neprvočíselné pole]
    Neprvočíselné pole je konečné pole $(F, +, \cdot)$, kde $|F| = p^k$, $p$ je prvočíslo a
    zároveň $k > 1, k \in \mathbb{N}$. Tedy všechny případy, kdy pro $k$ z věty \ref{the:field}
    platí, že $k>1$.
\end{definition}

V případě neprvočíselných polí nebude fungování operací tak zřejmé jako tomu bylo u
prvočíselných polí. Použitím stejného triku
jako u prvočíselných polí, tedy použití běžných operací $+$ a $\cdot$ modulo počet prvků,
totiž nejsme schopni vytvořit pole. Problém je v takovém případě operace $\cdot$, kdy
pouze s přidáním modula nebude splňovat požadované vlastnosti z definice pole\ref{def:field}.

Opět platí, že prvky pole můžeme označit jak chceme, ale je dobré, udělat to tak, aby se nám
s nimi vhodně pracovalo. V případě neprvočíselných polí je pro jejich odvození vhodné
označit si prvky jako polynomy v proměnné $t$, kde koeficienty jsou z $\mathbb{F}_p$ až do
stupně $k - 1$, kde $n = p^k$ pro $\mathbb{F}_n$.

\begin{example}[Definice pro $\mathbb{F}_4$]
    $$4 = 2^2,\; p = 2,\; k = 2$$
    Polynomy v tomto případě tedy budou:
    \begin{table}[h]
        \centering
        \begin{tabular}{|c|c|c|c|c|}
        \hline
        Polynomy          & $0$ & $1$ & $t$ & $t + 1$ \\ \hline
        Pomyslná hodnota & $0$ & $1$ & $2$ & $3$      \\ \hline
        \end{tabular}
        \caption{Vyjádření polynomů pro $\mathbb{F}_{4}$}
        \label{tab:F4_pol}
        \end{table}
\end{example}

Pro vytvoření aditivní operace stačí sčítat polynomy v každém stupni modulo $p$.

\begin{example}[Tvorba aditivní operace pro $\mathbb{F}_4$]
Budeme sčítat polynomy v každém stupni modulo $p$
\[
2 + 3 = t + (t + 1)
=
\begin{array}{rr}
    t & + 0\\
    t & + 1\\ \hline
    0 & + 1\\
      &
\end{array}
= 1
\]

\[
1 + 1 = 1 + 1 = t
=
\begin{array}{rr}
    0t & + 1\\
    0t & + 1\\ \hline
    0t & + 0\\
       &
\end{array}
= 0
\]
\[
    1 + 2 = 1 + t
=
\begin{array}{rr}
    0t & + 1\\
    t  & + 0\\ \hline
    t  & + 1\\
       &
\end{array}
= 3
\]
Stejným postupem pro ostatní hodnoty (některé jdou rovnou doplnit díky vlasnostem operace $+$)
dostaneme kompletní tabulku definujicí aditivní operaci $+$.

\begin{table}[h]
    \centering
    \begin{tabular}{|l|l|l|l|l|}
    \hline
    \multicolumn{1}{|c|}{$+$} & \multicolumn{1}{c|}{$0$} & \multicolumn{1}{c|}{$1$} & \multicolumn{1}{c|}{$2$} & \multicolumn{1}{c|}{$3$} \\ \hline
    \multicolumn{1}{|c|}{$0$} & \multicolumn{1}{c|}{$0$} & \multicolumn{1}{c|}{$1$} & \multicolumn{1}{c|}{$2$} & \multicolumn{1}{c|}{$3$} \\ \hline
    $1$                       & $1$                      & $0$                      & $3$                      & $2$                      \\ \hline
    $2$                       & $2$                      & $3$                      & $0$                      & $1$                      \\ \hline
    $3$                       & $3$                      & $2$                      & $1$                      & $0$                      \\ \hline
    \end{tabular}
    \caption{Aditivní operace pro $\mathbb{F}_{4}$}
    \label{tab:F4_plus}
    \end{table}

\end{example}

\begin{example}[Příklad polynomů pro $\mathbb{F}_{125}$]
    $$125 = 5^3, \; p = 5, \; k = 3$$
    Polynomy budou následující:
    $$0, \; 1, \; 2, \; 3, \; 4, \; t, \; t + 1, \; t + 2, \; t + 3, \; t + 4,
        2t + 1,\; \dots \;4t^2 + 4t + 3, \;  4t^2 + 4t + 4$$

Ukázka součtu dvou polynomů:
\[
    (4t + 2) + (t^2 + 2t + 3)
    =
    \begin{array}{rrr}
        0t^2 & + 4t & + 2\\
        t^2  & + 2t & + 3\\ \hline
        t^2  & + 1t & + 0\\
          &
    \end{array}
= t^2 + t
\]
\end{example}

Při vytváření multiplikativní operace se nám stane, že po vynásobení dvou polynomů
vznikne polynom stupně, který je větší, než $k - 1$ a tedy není mezi polynomy daného pole.
Budeme proto potřebovat tzv. redukční polynom.
\begin{definition}[Redukční polynom]
    Redukční polynom $P_{red}:$ polynom stupně $k$, který je nerozložitelný na součin
    polynomů stupně nižších (řekneme, že je ireducibilní).
\end{definition}


\begin{example}[Hledání redukčního polynomu pro $\mathbb{F}_4$]

Všechny polynomy stupně $k=2$:
    \begin{itemize}
        \item $t^2$ lze rozložit na $t \cdot t$
        \item $t^2 + 1$ lze rozložit na $(t + 1) \cdot (t + 1)$
        \item $t^2 + t$ lze rozložit na $t \cdot (t + 1)$
        \item $t^2 + t + 1$ nelze rozložit
    \end{itemize}

\[
    (t + 1) \cdot (t + 1) =
    \begin{array}{rrr}
        &t&+1 \\
        &t&+1 \\ \hline
        &t&+1 \\
        t^2&+t& \\ \hline
        t^2&+0t&+1
    \end{array}
= t^2 + 1
\]
\end{example}

Tvorba multiplikativní operace: po vynásobení dvou prvků z $\mathbb{F}_{p^k}$ vyjádřených pomocí
polynomů odečítáme (je-li třeba) $t^h \cdot P_{red}$ tak dlouho, až je výsledek stupně nejvýše
$k - 1$.

\begin{example}[Aplikace multiplikativní operace v $\mathbb{F}_4$ a využití $P_{red}$]
$$t \cdot (t + 1) = t^2 + t$$
Polynom $t^2 + t$ má ale příliš vysoký stupeň (vyšší, než $k - 1$). Začneme proto s odečítáním
redukčního polynomu\footnote{Můžeme odečítat i jeho $t^h$ násobky, ale v tomto případě stačí
redukční polynom sám o sobě.}, který je v tomto případě $t^2 + t + 1$.
\[
    (t^2 + t) - (t^2 + t + 1)
    =
    \begin{array}{rrr}
        t^2&+t&+0 \\
       -(t^2&+t&+1) \\ \hline
        0t^2&+0t&+1 \\
        &&
    \end{array}
    = 1
\]
\end{example}

\begin{example}[Tvorba tabulky multiplikativní operace v $\mathbb{F}_4$]
    Hodnoty pro $0$ a $1$ jsou jasné. V předchozím příkladu jsme spočítali,
    že $3 \cdot 2 = 1$, díky čemuž zároveň víme že, $2 \cdot 3 = 1$. Ostatní
    hodnoty jsme již schopni doplnit díky požadovaným vlastnostem operace $\cdot$.
    Ale pojďme ověřit $2 \cdot 2$.

    $$2 \cdot 2 = t \cdot t = t^2$$
    Stupeň polynomu je větší, než $k - 1$. Odečteme $T_{red}$.
    \[
        t^2 - (t^2 + t + 1)
        =
        \begin{array}{rrr}
            t^2&+0t&+0 \\
           -(t^2&+t&+1) \\ \hline
            0t^2&+t&+1 \\
            &&
        \end{array}
        = t + 1 = 3
    \]

    \begin{table}[h]
        \centering
        \begin{tabular}{|c|c|c|c|c|}
        \hline
        $\cdot$ & $0$ & $1$ & $2$ & $3$ \\ \hline
        $0$     & $0$ & $0$ & $0$ & $0$ \\ \hline
        $1$     & $0$ & $1$ & $2$ & $3$ \\ \hline
        $2$     & $0$ & $2$ & $3$ & $1$ \\ \hline
        $3$     & $0$ & $3$ & $1$ & $2$ \\ \hline
        \end{tabular}
        \caption{Multiplikativní operace pro $\mathbb{F}_{4}$}
        \label{tab:F4_mul}
    \end{table}
\end{example}

\subsection{Konstrukce množiny reálných čísel}
Využijeme definici reálných čísel pomocí Dedekindových řezů.

\begin{definition}[Dedekinduv řez]
    Dedekindův řez $D$ je podmnožina racionálních čísel $D \subseteq \mathbb{Q}$, která splňuje:
    \begin{enumerate}
        \item $x \in D \Rightarrow \exists y > x, y \in D$ \hfill Neexistence největšího prvku
        \item $x \in D, y < x \Rightarrow y \in D$ \hfill
    \end{enumerate}
\end{definition}

Příklady Dedekindových řezů:
\begin{itemize}
    \item $\mathbb{Q}$ tento řez označme $\infty$
    \item $\emptyset$ tento řez označme $- \infty$
    \item $\mathbb{Q}^{-}$
    \item $\{x \in \mathbb{Q};\,x < 7\}$
    \item $\{x \in \mathbb{Q};\, x \cdot x < 2 \vee x < 0\}$
\end{itemize}

Budeme li uvažovat všechny dedekindovy řezy, dostaneme množinu rozšířených reálných
čísel, kterou budeme označovat $\overline{\mathbb{R}}$.

Potom $$\mathbb{R} = \overline{\mathbb{R}} \smallsetminus \{-\infty, \infty\}$$
Kde $\mathbb{R}$ označuje množinu reálných čísel.

\begin{definition}[Součet Dedekindových řezů]
    $$D + E = \{x + y; x\in D, y \in E\}$$
\end{definition}

\begin{definition}[Nezáporný dedekindův řez]
    Řekneme, že dedekindův řez $D$ je nezáporný právě tehdy, když:
    $$D \supseteq \mathbb{Q}^{-}$$
\end{definition}

\begin{definition}[Součin Dedekindových řezů]
    Předpokládáme, že řezy $D$ a $E$ jsou nezáporné.
    $$D \cdot E = \{x \cdot y; \forall x, y \geq 0, x \in D, y \in E\} \cup \{z; z < 0, z \in \mathbb{Q}\}$$
    Pokud je jeden z řezů záporný a druhý nezáporný, potom musíme definovat opačný řez,
    k zápornému řezu vyrobit řez opačný, použijeme násobení nezáporných řezů a z výsledku opět vyrobíme
    řez opačný.

    Pokud budou oba řezy záporné, z obou řezů vezmu opačné řezy, použiji násobení nezáporných řezů
    a dostanu korektní výsledek. \footnote{Násobení dedekindových řezů bylo na přednášce definováno
    pouze takto částečně.}
\end{definition}

\subsubsection*{Komplexní čísla}
Uvažujme $\mathbb{R} \times \mathbb{R} = \mathbb{R}^2$. V $\mathbb{R}^2$ není
definována multiplukativní operace $(a, b) \cdot (c, d)$. Pokud v $\mathbb{R}^2$
multiplikativní operaci definujeme takovým způsobem, aby splňovala
vlastnosti na multiplikativní operaci z definice pole\ref{def:field}, dostáváme:

$$(a, b) \cdot (c, d) = (ac - bd, ad + bc)$$

Což je totéž jako
$$(a + bi) \cdot (c + di) = ac + (b + d)i^2 + a \cdot di + bi \cdot c = ac - bd + (ad + bc)i, \; i^2 = -1$$

Přidáním této operace dostaneme množinu komplexních čísel $\mathbb{C}$, která opět tvoří
strukturu pole.

Byly snahy tento postup zobecnit. Pro $\mathbb{R}^3$ avšak vhodná multiplikativní operace, která by
vyhovovala požadavkům z definice pole\ref{def:field} neexistuje.

Pro $\mathbb{R}^4$ už multiplikativní operaci splňujicí požadované vlastnosti vytvořit
lze, tím dostáváme tzv. kvaterniony, značíme je $\mathbb{H}$.
Opět máme \uv{pomůcky} a pravidla pro jejich násobení.
Kvaterniony zapisujeme ve tvaru:
$$a + bi + cj + dk$$
A pravidla pro jejich násobení jsou:
\begin{enumerate}
    \item $i^2 = j^2 = k^2 = -1$
    \item $ij = -ji = k$
\end{enumerate}
U kvaternionů však máme jednu změnu, nejedná se o komutativní pole (je to zřejmé z druhého pravidla)
a jsou tedy prvním příkladem nekomutativního pole se kterým jsme se v přednáškách zatím setkali.

\subsection{Mohutnosti nekonečných množin}
Kardinalita nejvšednější nekonečné množiny, přirozených čísel, je definována jako \uv{alef 0}
$$|\mathbb{N}| = \aleph_0$$
Jakákoliv jiná nekonečná množina bude mít stejnou kardinalitu, pokud existuje bijekce
mezi touto nekonečnou množinou a množinou přirozených čísel.

\begin{table}[h]
    \centering
    \begin{tabular}{cccccccccccc}
    $\mathbb{N}$   & $1$ & $2$ & $3$  & $4$ & $5$  & $6$  & $7$  & $8$  & $9$  & $10$ & $\dots$ \\
    $\mathbb{N}_0$ & $0$ & $1$ & $2$  & $3$ & $4$  & $5$  & $6$  & $7$  & $8$  & $9$  & $\dots$ \\
    $2\mathbb{N}+1$   & $1$ & $3$ & $5$  & $7$ & $9$  & $11$ & $13$ & $15$ & $17$ & $19$ & $\dots$ \\
    $\mathbb{Z}$  & $0$ & $1$ & $-1$ & $2$ & $-2$ & $3$  & $-3$ & $4$  & $-4$ & $5$  & $\dots$ \\
    $\mathbb{Q}$ &
      $\frac{0}{1}$ &
      $\frac{-1}{1}$ &
      $\frac{-2}{1}$ &
      $\frac{-1}{2}$ &
      $\frac{1}{2}$ &
      $\frac{2}{1}$ &
      $\frac{-3}{1}$ &
      $\frac{-1}{3}$ &
      $\frac{1}{3}$ &
      $\frac{3}{1}$ &
      $\dots$
    \end{tabular}
    \caption{Ukázka některých bijekcí s přirozenými čísly}
    \label{tab:Naturlas_bijection}
\end{table}

Z bijekcí naznačených v tabulce \ref{tab:Naturlas_bijection} vyplývá:
$$|\mathbb{N}| = |\mathbb{N}_0| = |2\mathbb{N}+1| = |\mathbb{Z}| = |\mathbb{Q}| = \aleph_0$$

\subsubsection*{Mohutnost množiny reálných čísel}
Mohutnost množiny reaálných čísel je větší, než mohutnost množiny celých čísel.
$$|\mathbb{R}| > |\mathbb{N}|$$
\begin{proof}
    Neexistence bijekce mezi $\mathbb{R}$ a $\mathbb{N}$

    Předpokládejme, že bijekce mezi $\mathbb{R}$ a $\mathbb{N}$ existuje.
    Vezměme reálný interval $\langle0, 1)$ a předpokládejme, že jeho prvky lze seřadit\footnote{Tento
    předpoklad vychází z předpokladu, že existuje bijekce s $\mathbb{N}$.}.

    Za předpokladu, že jsme schopni hodnoty tohoto intervalu seřadit, jsme schopni
    je všechny reprezentovat nekonečnou tabulkou \ref{tab:diag_real}.


\begin{table}[]
    \centering
    \begin{tabular}{cccccl}
    $a_1 = $ & $0,$     & \cellcolor[HTML]{3166FF}$a_{11}$ & $a_{12}$                         & $a_{13}$                         & $\dots$  \\
    $a_2 = $ & $0,$     & $a_{21}$                         & \cellcolor[HTML]{3166FF}$a_{22}$ & $a_{23}$                         & $\dots$  \\
    $a_3 = $ & $0,$     & $a_{31}$                         & $a_{32}$                         & \cellcolor[HTML]{3166FF}$a_{33}$ & $\dots$  \\
    $\vdots$ & $\vdots$ & $\vdots$                         & $\vdots$                         & $\vdots$                         & $\ddots$
    \end{tabular}
    \caption{Seřazení hodnot reálného intervalu $\langle 0, 1)$}
    \label{tab:diag_real}
\end{table}

    Teď vytvoříme číslo $b = 0,b_1 b_2 b_3 \ldots$, kde každou číslici $b_i$ určíme následovně:
    \[
    b_i =
    \left\{
    \begin{array}{ll}
        1 & \text{pokud} \; a_{ii} \neq 1\\
        2 & \text{pokud} \; a_{ii} = 1   \\
    \end{array}
    \right.
\]

    Tím jsme ale zkonstruovali reálné číslo $b$, které se liší\footnote{A to alespoň v
    jedné číslici na diagonále (zobrazeno modře).} od každého čísla v tabulce
    \ref{tab:diag_real}.

    Z našich předpokladů však vycházelo, že v tabulce musí být obsažena všechna čísla z daného
    intervalu. Dostáváme tedy spor a z toho vychází, že naše předpoklady nebyly správné a
    neexistuje bijekce mezi $\mathbb{N}$ a reálným intervalem $\langle 0, 1)$. Tím pádem nemůže
    existovat bijekce ani mezi $\mathbb{R}$ a $\mathbb{N}$.

    Z důkazu nám zároveň vychází, že $|\mathbb{R}| > |\mathbb{N}|$.
\end{proof}

Kardinalitu reálných čísel budeme značit $c$
$$|\mathbb{R}| = c > \aleph_0$$

\begin{definition}[Kardinalita potenčních množin přirozených čísel]
    Značíme pomocí $\aleph_i$
    $$|P(\mathbb{N})| = \aleph_1$$
    $$|P(P(\mathbb{N}))| = \aleph_2$$
    $$|P(P(P(\mathbb{N})))| = \aleph_3$$
    $$\vdots$$
    Kde
    $$\aleph_0 < \aleph_1 < \aleph_2 < \aleph_3 \ldots$$
\end{definition}

\begin{proof}
    Neexistence bijekce mezi $\mathbb{N}$ a $P(\mathbb{N})$

    Předpokládejme, že $f: \mathbb{N} \rightarrow P(\mathbb{N})$ je bijekce.

    Nyní uvažujme množinu
    $$D = \{n \in \mathbb{N}; n \notin f(n)\}$$
    $D$ je nějaká podmnožina všech přirozených čísel $a$, kde bijekce $f$ zobrazí $a$ na
    podmnožinu, která číslo $a$ neobsahuje.

    Vzhledem k tomu, že $D \subseteq \mathbb{N}$, musí platit $D \in P(\mathbb{N})$, pak
    $$\exists m \in \mathbb{N}: f(m) = D$$
    Potom ale $$m \in D \Leftrightarrow m \notin D$$
    Čímž se dostáváme ke sporu a bijekce $f$ jejíž existenci jsme předpokládali neexistuje.
\end{proof}

Není jednoznačné, zda $\aleph_1 = c$ \footnote{Jedná se o nezávislý axiom.}.

\subsubsection*{Mohutnost množiny komplexních čísel}
$$|\mathbb{C}| = |\mathbb{R}^2| = |\mathbb{R}| = c$$
Což ovšem znamená, že musíme být schopni najít bijekci mezi $\mathbb{R}$ a $\mathbb{R}^2$.

Toho dosáhneme následovně, každé reálné číslo zobrazíme na uspořádanou dvojici takto:
$$0,3451239956\ldots \rightarrow (0,35295\ldots ;0,41396\ldots)$$

Tímto způsobem jsme schopni obecně najít bijekci mezi $\mathbb{R}$ a $\mathbb{R}^n$.
