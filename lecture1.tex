\section{První přednáška}
Matematika se historicky rodila jako celá řada disciplín, například jak tady máme jednu
z nich, lineární algebra, nebo matematická analýza, teorie čísel a řada dalších...

Byly to takové historicky izolované disciplíny, v podobné situaci je dnes fyzika, ta
má dnes také různé disciplíny a nedaří se je fyzikům spojit, i když se o to už
dlouho pokoušejí a hledají tzv. teorii všeho. V matematice nějaká taková teorie všeho již
byla nalezena a máme tu výhodu, že se podařilo najít ten \uv{spojovací materiál},
kterým je logika a teorie množin.

Díky tomu se matematiku podařilo dostat na společný jazyk, takže ať už studujete
teorii pra\-vdě\-po\-dob\-no\-sti, nebo matematickou analýzu, tak začne se s nějakou
definicí, co je jakási množina a tak dále...

\subsection{Logika}
V logice je základním pojmem výrok, neboli tvrzení, o kterém můžeme říct, zda je pravdivé,
nebo nepravdivé. Tedy přiřazujeme mu pravdivostní hodnotu, nulu, nebo jedničku. Výroky následně
rozlišujeme na jednoduché (atomární), které nejdou dále rozložit a na výroky složité, které
jsou pospojovány logickými spojkami. Nejznámější logické spojky jsou tyto: $\wedge$, $\vee$,
$\rightarrow$, $\leftrightarrow$. Zdaleka se však nejedná o jejich vyčerpávajícíc výčet.
Uvědomme si, že existuje $2^4$ různých logických spojek (binárních).

Další nad čím je vhodné se z těchto základů pozastavit je pojem výroku. Ne všechno v běžné řeči
je výrok. Je nutné si uvědomit, co výrok je a co není. A i když něco výrokem je, tak to
ještě nezmanená, že jsme schopni určit pravdivostní hodnotu tohoto výroku. Například tvrzení:
\textit{Na Saturnově měsíci je voda}, určitě se jedná o výrok, ale jeho pravdivostní hodnotu neznáme.
Výrokem ale určitě nejsou různá zvolání, výkřiky, otázky...

Je každá dobře utvořená oznamovací věta výrokem? Není, například věta \textit{Colorless green ideas
sleep furiously.} je z mluvnického hlediska utvořena správně a jedná se o oznamovací větu, ale
významově je to takový nesmysl, že nelze určit zda se jedná o výrok, protože tomu nelze přiřadit
pravdivostní hodnota a to ani hypotetická.

Pomocí těchto výroků a spojek se snažíme dokazovat věty. V matematice máme 4 základní kameny:
\begin{itemize}
	\item Primitivní pojmy: pojmy které se nedefinují a nevysvětlují, například bod.
	\item Definice: zavádějí pojmy, pomocí pojmů již známých, například definice prvočísla.
	\item Axiomy: tvrzení, která se nedokazují a která se považují za platná.
	\item Vety: tvrzení, které se musí dokázat a odvozuje se z tvrzení již známých.
\end{itemize}

\subsubsection*{Důkazové techniky}
\begin{itemize}
	\item Důkaz přímý: ukážeme, že $v \rightarrow w$.
	\item Důkaz nepřímy: ukážeme, že $\lnot w \rightarrow v$.
	\item Důkaz sporem:
\end{itemize}

\subsection{Teorie množin}

Množina je primitivní pojem. I přesto, že se jedná o primitivní pojem, budeme si ho nějakým
způsobem specifikovat, aby si každý z nás pod tímto primitivním pojmem představil to stejné.

\begin{concept}[Množina]
Množina je nějaký soubor prvků, které se neopakují.
Nad množinami jsou opět definovány nějaké operace, jako sjednocení, průnik, doplněk...
Množiny a operace nad nimi můžeme vizualizovat například pomocí vennových diagramů.
\end{concept}

\subsubsection*{Mohutnost}
\begin{concept}[Mohutnost množiny]
    Mohutnost množiny jednoduše udává počet jejich prvků. Mohutnost množiny $A$ se značí
    jako $|A|$, nebo jako $card(A)$.
    Mohutnost prázdné množiny je 0, $|\emptyset| = 0$.
\end{concept}
\begin{example}[Mohutnost množiny]
    Nechť $A = \{1, 2, 3\}$. Potom $|A| = card(A) = 3$.
\end{example}

\begin{definition}[Mohutnost množiny přirozených čísel]
    $|\mathbb{N}|=\aleph_0$.
\end{definition}

\subsubsection*{Relace}
Před zavedením relace je nutné nejprve definovat pojem kartézského součinu množin.
\begin{definition} [Kartézský součin]
	Kartézský součin dvou množin se skládá z uspořádaných dvojic.
	$$A \times B = \{(a, b)\: | \: a \in A, b \in B\}$$
\end{definition}
\begin{example}[Kartézský součin]
    \[A = \{u, v\}
	\textrm{, }
	B = \{1, 2, 3\}\]
	$$A \times B = \{(u, 1), (u, 2), (u, 3), (v, 1), (v, 2), (v, 3)\}$$
\end{example}

\begin{definition}[Relace]
    Relace je libovolná podmnožina kartézského součinu.
    $$R \subseteq A \times B \textrm{, například: } R = \{(u, 1), (u, 2), (v, 1)\}$$
\end{definition}

Relaci lze vyjádřit i graficky jako orientovaný graf.

Některé binární relace jsou zobrazení, neboli funkce (funkce tomu říkáme tehdy, když je cílová
množina číselná).

% definice
\begin{definition}[Zobrazení]
	Řekneme, že relace $R \subseteq A \times B$ je zobrazení jestliže:
	$$(a, b_1) \in \mathbb{R} \wedge (a, b_2) \in \mathbb{R} \Rightarrow b_1 = b_2$$
\end{definition}

Zobrazení můžeme značit jako $f: A \rightarrow B$


Inverzní relace, stejná jako relace R, jen v grafické reprezentaci otočíme šipky na druhou stranu.
$R^{-1} \subseteq B \times A,  (b, a) \in R^{-1} \Leftrightarrow (a, b) \in \mathbb{R}$

Řekneme, že zobrazení $f: A \rightarrow B$ je injekce (prosté zobrazení)
jestliže $f(a_1) = b \wedge f(a_2) = b \Rightarrow a_1 = a_2$

Definiční obor $D(f) = Dom(f) = \{a \in A; \exists b \in B, f(a) = b\}$

Obor hodnot $H(f) = Im(f) = \{b \in B; \exists a \in A, f(a) = b\}$

%definice
Zobrazení je surjekce (zobrazení na), jestliže oborem hodnot je celá cílová množina, tedy $H(f) = B$.

Jestliže je zobrazení surjekce a současně injekce, jedná se o bijekci. Za bijekci se navíc velmi
často považuje zobrazení, které je bijekce a zároveň $D(f) = A$.
Potom platí, že pokud existuje bijekce mezi konečnými množinami $A$ a $B$, potom $|A| = |B|$.

Jestliže zobrazení $f$ je injekce, potom inverzní relace je opět zobrazení (a opet injekce).

\subsubsection*{Binární relace na množinách}
Relací na množině se rozumí binární relace, kde jsou oba prvky kartézského součinu tatáž
množina, tedy $R \subseteq A \times A$.
Specialní případy:
% definice
\begin{itemize}
	\item Reflexivní relace: $(a, a) \in R, \forall a \in A$.
	\item Symetrická relace: $(a, b) \in R \Rightarrow (b, a) \in R$.
	\item Antisymetrická relace: $(a, b) \in R \wedge (b, a) \in R \Rightarrow a = b$.
	\item Tranzitivní relace: $(a, b) \in R \wedge (b, c) \in R \Rightarrow (a, c) \in R$.
	\item Ireflexivní relace: $(a, a) \notin R, \forall a \in A$.
\end{itemize}
Relace, která je současně reflexivní, symetrická a tranzitivní se nazývá ekvivalence.
Relace ekvivalence vždy rozdělí původní množinu na podmnožiny, kterým říkáme třídy ekvivalence,

Relace, která je reflexivní, antisymetrická a tranzitivní se nazývá relace uspořádání a vytváří
POSET (Partialy Ordered SET).

\subsubsection*{Opeace}
Operace je obecně zobrazení, konkrétní podoba tohoto zobrazení záleží na aritě operace.

Binární operace je zobrazení z kartézského součinu dvou množin do nějaké další množiny, velmi často
jsou všechny tyto 3 množiny totožné.
$$f: A \times B \rightarrow C$$

\subsubsection*{Konstrukce přirozených čísel}
Z axiomů teorie množin víme, že prázdná množina existuje. Definujeme unární operaci
následníka:
$$A' = A \cup \{A\}$$
Opakovanou aplikací operace následníka na původně prázdnou množinu jsme schopni vytvořit
všechna přirozená čísla.
$$\emptyset' = \emptyset \cup \{\emptyset\} = \{\emptyset\}$$
$$\{\emptyset\}' = \{\emptyset\} \cup \{\{\emptyset\}\} = \{\emptyset, \{\emptyset\}\}$$

Definice operace plus ($+$) na takto definovaných přirozených číslech:
$$A + \emptyset = A$$
$$A + B' = (A+B)'$$

Definice operace krát ($cdot$) na takto definovaných přirozených číslech:
$$A \cdot \emptyset = \emptyset$$
$$A \cdot B' = A \cdot B + A$$

\subsubsection*{Konstrukce přirozených čísel}
\subsubsection*{Konstrukce celých čísel}
Oproti přirozeným číslům musíme přidat nulu a záporné hodnoty. Můžeme k tomu
využít relaci ekvivalence. Uvažujme dvojice přirozených čísel
$(a, b) \in \mathbb{N} \times \mathbb{N}$. Řekneme, že $(a, b) \sim (c, d): a + d = b + c$

Věta: výše definovaná relace $\sim$ je relace ekvivalence.

Důkaz:
\begin{itemize}
	\item Reflexivita $(a, b) \sim (a, b): a + b = b + a$
	\item Symetrie $(a, b) \sim (c, d) \Rightarrow (c, d) \sim (a, b): a + d = b +
	c \Rightarrow c + b = d + a$
	\item Reflexivita $(a, b) \sim (c, d) \wedge (c, d) \sim (e, f) \Rightarrow (a, b) \sim (e, f):
	a + d = b + c \wedge c + f = d + e \Rightarrow a + f = b + e$
\end{itemize}
Množinu $\mathbf{N} \times \mathbf{N}$ tedy relace $\sim$ rozdělí na třídy rozkladu, kde každá třída
bude tvořena uspořádanými dvojicemi, se stejným rozdílem. Tyto třídy ekvivalence mohou reprezentovat
všechna celá čísla.

\subsubsection*{Konstrukce racionalních čísel}
Racionální čísla narozdíl od celých a přirozených mají lepší vlastnosti, umožňují dělení.
Racionální čísla jsou první obor, který se nazývá těleso, nebo také pole, pole/těleso racionálních
čísel.

Uvažujme uspořádané dvojice celých čísel
$(a, b) \in \mathbf{Z} \times \mathbf{Z} \smallsetminus \{0\}$.
Na takovýchto dvojicích zavedeme následujicí relaci: $(a, b) \sim (c, d): a \cdot d=b \cdot c$.
Kde $\cdot$ představuje násobení nad množinou celých čísel.

Opět tvrdíme, že relace $\sim$ je relace ekvivalence a zase množinu
$\mathbf{Z} \times \mathbf{Z} \smallsetminus \{0\}$ rozdělí na třídy ekvivalence, kde jednotlivé
třídy mohou reprezentovat všechna racionální čísla.