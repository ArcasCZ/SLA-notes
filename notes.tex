\documentclass[a4paper, 11pt]{article}
\usepackage[left=2cm,text={17cm, 24cm},top=3cm]{geometry}
\usepackage[czech]{babel}
\usepackage[utf8]{inputenc}
\usepackage[unicode]{hyperref}
\usepackage{url}
\usepackage{amssymb}
\usepackage{amsthm}
\usepackage{amsmath}
\usepackage{pxfonts}
\usepackage[parfill]{parskip}
\usepackage{hyphenat}
\usepackage{enumitem}
\usepackage[table,xcdraw]{xcolor}

\newtheoremstyle{break}
  {\topsep}{\topsep}%
  {\itshape}{}%
  {\bfseries}{}%
  {\newline}{}%
\theoremstyle{break}

\newtheorem*{concept}{Primitivní pojem}
\newtheorem{definition}{Definice}
\newtheorem{axiom}{Axiom}
\newtheorem{theorem}{Věta}
\newtheorem*{example}{Příklad}


\begin{document}

%-----------------------------------------------------------------------------------
\begin{titlepage}

	\begin{center}
		\textsc{\Huge Vysoké učení technické v~Brně}\\
				  \huge{\textsc{Fakulta informačních technologií}}\\
		\vspace{\stretch{0.382}}
		{\LARGE 	Lineární algebra\,--\, SLA}\\
				{\Huge Poznámky z přednášek}
		\vspace{\stretch{0.618}}
	\end{center}
	{\Large\today \hfill David Sedlák (xsedla1d@stud.fit.vutbr.cz)}

	\begin{center}
		Pullrequesty a připomínky můžete směřovat do repozitáře: \url{www.github.com/Dajvid/SLA-notes}
	\end{center}
\end{titlepage}
%-----------------------------------------------------------------------------------

\tableofcontents

\section{První přednáška}
Matematika se historicky rodila jako celá řada disciplín, například jak tady máme jednu
z nich, lineární algebra, nebo matematická analýza, teorie čísel a řada dalších...

Byly to takové historicky izolované disciplíny, v podobné situaci je dnes fyzika, ta
má dnes také různé disciplíny a nedaří se je fyzikům spojit, i když se o to už
dlouho pokoušejí a hledají tzv. teorii všeho. V matematice nějaká taková teorie všeho již
byla nalezena a máme tu výhodu, že se podařilo najít ten \uv{spojovací materiál},
kterým je logika a teorie množin.

Díky tomu se matematiku podařilo dostat na společný jazyk, takže ať už studujete
teorii pra\-vdě\-po\-dob\-no\-sti, nebo matematickou analýzu, tak se začne s nějakou
definicí, co je jakási množina a tak dále...

\subsection{Logika}
V logice je základním pojmem výrok, neboli tvrzení, o kterém můžeme říct, zda je pravdivé,
nebo nepravdivé. Tedy přiřazujeme mu pravdivostní hodnotu, nulu, nebo jedničku. Výroky následně
rozlišujeme na jednoduché (atomární), které nejdou dále rozložit a na výroky složité, které
jsou pospojovány logickými spojkami. Nejznámější logické spojky jsou tyto: $\wedge$, $\vee$,
$\rightarrow$, $\leftrightarrow$. Zdaleka se však nejedná o jejich vyčerpávajícíc výčet.
Uvědomme si, že existuje $2^4$ různých logických spojek (binárních).

Další nad čím je vhodné se z těchto základů pozastavit je pojem výroku. Ne všechno v běžné řeči
je výrok. Je nutné si uvědomit, co výrok je a co není. A i když něco výrokem je, tak to
ještě nezmanená, že jsme schopni určit pravdivostní hodnotu tohoto výroku. Například tvrzení:
\textit{Na Saturnově měsíci je voda}, určitě se jedná o výrok, ale jeho pravdivostní hodnotu neznáme.
Výrokem ale určitě nejsou různá zvolání, výkřiky, otázky...

Je každá dobře utvořená oznamovací věta výrokem? Není, například věta \textit{Colorless green ideas
sleep furiously.} je z mluvnického hlediska utvořena správně a jedná se o oznamovací větu, ale
významově je to takový nesmysl, že nelze určit, zda se jedná o výrok, protože tomu nelze přiřadit
pravdivostní hodnota a to ani hypotetická.

Pomocí těchto výroků a spojek se snažíme dokazovat věty. V matematice máme 4 základní kameny:
\begin{itemize}
	\item Primitivní pojmy: pojmy které se nedefinují a nevysvětlují, například bod.
	\item Definice: zavádějí pojmy, pomocí pojmů již známých, například definice prvočísla.
	\item Axiomy: tvrzení, která se nedokazují a která se považují za platná.
	\item Vety: tvrzení, které se musí dokázat a odvozují se z tvrzení již známých.
\end{itemize}

\subsection{Teorie množin}

Množina je primitivní pojem. I přesto, že se jedná o primitivní pojem, budeme si ho nějakým
způsobem specifikovat, aby si každý z nás pod tímto primitivním pojmem představil to stejné.

\begin{concept}[Množina]
Množina je nějaký soubor prvků, které se neopakují.
Nad množinami jsou opět definovány nějaké operace, jako sjednocení, průnik, doplněk...
Množiny a operace nad nimi můžeme vizualizovat například pomocí vennových diagramů.
\end{concept}

\subsubsection*{Mohutnost}
\begin{concept}[Mohutnost množiny]
    Mohutnost množiny jednoduše udává počet jejich prvků. Mohutnost množiny $A$ se značí
    jako $|A|$, nebo jako $card(A)$.
    Mohutnost prázdné množiny je 0, $|\emptyset| = 0$.
\end{concept}
\begin{example}[Mohutnost množiny]
    Nechť $A = \{1, 2, 3\}$. Potom $|A| = card(A) = 3$.
\end{example}

\begin{definition}[Mohutnost množiny přirozených čísel]
	Mohutnost množiny přirozených čísel definujeme jako \uv{alef nula}:
    $$|\mathbb{N}|=\aleph_0$$
\end{definition}

\subsubsection*{Relace}
Před zavedením relace je nutné nejprve definovat pojem kartézského součinu množin.
\begin{definition} [Kartézský součin množin]
	Kartézský součin dvou množin se skládá z uspořádaných dvojic.
	$$A \times B = \{(a, b)\: | \: a \in A, b \in B\}$$
\end{definition}
\begin{example}[Kartézský součin množin]
	Mejme dvě množiny $A$ a $B$:
    \[A = \{u, v\}
	\textrm{, }
	B = \{1, 2, 3\}\]
	Jejich kartézský součin $A\times B$ je potom:
	$$A \times B = \{(u, 1), (u, 2), (u, 3), (v, 1), (v, 2), (v, 3)\}$$
\end{example}

\begin{definition}[Relace]
    Relace je libovolná podmnožina kartézského součinu.
	$$R \subseteq A \times B \textrm{, například: } R = \{(u, 1), (u, 2), (v, 1)\}$$

	Relaci lze vyjádřit i graficky jako orientovaný graf.
\end{definition}


Některé binární relace jsou zobrazení, neboli funkce\footnote{funkce jim říkáme tehdy,
když je cílová množina číselná.}.

% definice
\begin{definition}[Zobrazení]
	Řekneme, že relace $R \subseteq A \times B$ je zobrazení jestliže:
	$$(a, b_1) \in \mathbb{R} \wedge (a, b_2) \in \mathbb{R} \Rightarrow b_1 = b_2$$
\end{definition}

Zobrazení můžeme značit jako $f: A \rightarrow B$

\begin{definition}[Inverzní relace]
	Uvažujme množiny $A$, $B$ a binární relaci $R \subseteq B \times A$, potom pro
	inverzní relaci k relaci $R$ platí:
	$$R^{-1} \subseteq B \times A,  (b, a) \in R^{-1} \Leftrightarrow (a, b) \in \mathbb{R}$$

	Inverzní relaci k relaci $R$ budeme značit jako $R^{-1}$

	Neformálně řečeno je inverzní relace k relaci $R$ stejná, jako relace R,
	jen v grafické reprezentaci otočíme šipky na druhou stranu.
\end{definition}


\begin{definition}[Definiční obor]
	$$D(f) = Dom(f) = \{a \in A; \exists b \in B, f(a) = b\}$$

	Budeme značit $D(f)$, nebo $Dom(f)$\footnote{Z anglického Domain.}.
\end{definition}

\begin{definition}[Obor hodnot]
	$$H(f) = Im(f) = \{b \in B; \exists a \in A, f(a) = b\}$$

	Budeme značit $H(f)$, nebo $Im(F)$\footnote{Z anglického Image.}.
\end{definition}

\begin{definition}[Injekce, prosté zobrazení]
	Řekneme, že zobrazení $f: A \rightarrow B$ je injekce (prosté zobrazení)
	jestliže: $$f(a_1) = b \wedge f(a_2) = b \Rightarrow a_1 = a_2$$

	Jestliže zobrazení $f$ je injekce,
	potom inverzní relace je opět zobrazení (a opět injekce).
\end{definition}

\begin{definition}[Surjekce]
	Uvažujme zobrazení $f: A \rightarrow B$

	Potom řekneme, že zobrazení $f$ je surjekce (zobrazení na),
	jestliže oborem hodnot je celá cílová množina, tedy právě tehdy, když:
	$$H(f) = B$$
\end{definition}

\begin{definition}[Bijekce]
	Uvažujme zobrazení $f: A \rightarrow B$

	Jestliže je zobrazení $f$ surjekce a současně injekce,
	řekneme, že se jedná o bijekci.\footnote{Za bijekci se navíc velmi
	často považuje zobrazení, které je bijekce a zároveň v platí $D(f) = A$.}

	Pokud existuje bijekce mezi konečnými množinami $A$ a $B$, potom:
	$$|A| = |B|$$.
\end{definition}

\subsubsection*{Binární relace na množinách}
Relací na množině se rozumí binární relace, kde jsou oba prvky kartézského součinu tatáž
množina, tedy $R \subseteq A \times A$.
Specialní případy:
% TODO should be in definition env.
\begin{itemize}
	\item Reflexivní relace: $(a, a) \in R, \forall a \in A$.
	\item Symetrická relace: $(a, b) \in R \Rightarrow (b, a) \in R$.
	\item Antisymetrická relace: $(a, b) \in R \wedge (b, a) \in R \Rightarrow a = b$.
	\item Tranzitivní relace: $(a, b) \in R \wedge (b, c) \in R \Rightarrow (a, c) \in R$.
	\item Ireflexivní relace: $(a, a) \notin R, \forall a \in A$.
\end{itemize}

\begin{definition}[Relace ekvivalence]
	Relace, která je současně reflexivní, symetrická a
	tranzitivní se nazývá relace ekvivalence.

	Relace ekvivalence vždy rozdělí původní množinu na disjunktní podmnožiny,
	kterým říkáme třídy ekvivalence,
\end{definition}

\begin{definition}[Relace uspořádání]
	Relace, která je reflexivní, antisymetrická a tranzitivní se nazývá
	relace uspořádání a vytváří POSET (Partialy Ordered SET).
\end{definition}


\subsubsection*{Opeace}
Operace je obecně zobrazení, konkrétní podoba tohoto zobrazení záleží na aritě operace.

Binární operace je zobrazení z kartézského součinu dvou množin do nějaké další množiny, velmi často
jsou všechny tyto 3 množiny totožné.
$$f: A \times B \rightarrow C$$

\subsubsection*{Konstrukce přirozených čísel}
Z axiomů teorie množin víme, že prázdná množina existuje. Definujeme unární operaci
následníka:
$$A' = A \cup \{A\}$$
Opakovanou aplikací operace následníka na původně prázdnou množinu jsme schopni vytvořit
všechna přirozená čísla.
$$\emptyset' = \emptyset \cup \{\emptyset\} = \{\emptyset\}$$
$$\{\emptyset\}' = \{\emptyset\} \cup \{\{\emptyset\}\} = \{\emptyset, \{\emptyset\}\}$$

Definice operace plus ($+$) na takto definovaných přirozených číslech:
$$A + \emptyset = A$$
$$A + B' = (A+B)'$$

Definice operace krát ($\cdot$) na takto definovaných přirozených číslech:
$$A \cdot \emptyset = \emptyset$$
$$A \cdot B' = A \cdot B + A$$

\subsubsection*{Konstrukce celých čísel}
Oproti přirozeným číslům musíme přidat nulu a záporné hodnoty. Můžeme k tomu
využít relaci ekvivalence.

Uvažujme dvojice přirozených čísel:
$$(a, b) \in \mathbb{N} \times \mathbb{N}$$
Potom řekneme, že:
$$(a, b) \sim (c, d): a + d = b + c$$

\begin{theorem}
	Výše definovaná relace $\sim$ je relace ekvivalence.
\end{theorem}
\begin{proof}
	Je třeba ověřit splnění vlastností, které z definice požadujeme od relace ekvivalence:
	\begin{itemize}
		\item $(a, b) \sim (a, b): a + b = b + a$ \hfill Reflexivita
		\item  $(a, b) \sim (c, d) \Rightarrow (c, d) \sim (a, b): a + d = b +
		c \Rightarrow c + b = d + a$ \hfill Symetrie
		\item $(a, b) \sim (c, d) \wedge (c, d) \sim (e, f) \Rightarrow (a, b) \sim (e, f):
		a + d = b + c \wedge c + f = d + e \Rightarrow a + f = b + e$ \hfill Tranzitivita
	\end{itemize}
\end{proof}

Množinu $\mathbb{N} \times \mathbb{N}$ tedy relace $\sim$ rozdělí na třídy rozkladu, kde každá třída
bude tvořena uspořádanými dvojicemi, se stejným rozdílem. Tyto třídy ekvivalence mohou reprezentovat
všechna celá čísla.

\subsubsection*{Konstrukce racionalních čísel}
Racionální čísla narozdíl od celých a přirozených mají lepší vlastnosti, umožňují dělení.
Racionální čísla jsou první obor, který se nazývá těleso, nebo také pole, pole/těleso racionálních
čísel.

Uvažujme uspořádané dvojice celých čísel
$(a, b) \in \mathbf{Z} \times \mathbf{Z} \smallsetminus \{0\}$.
Na takovýchto dvojicích zavedeme následujicí relaci: $(a, b) \sim (c, d): a \cdot d=b \cdot c$.
Kde $\cdot$ představuje násobení nad množinou celých čísel.

Opět tvrdíme, že relace $\sim$ je relace ekvivalence a zase množinu
$\mathbf{Z} \times \mathbf{Z} \smallsetminus \{0\}$ rozdělí na třídy ekvivalence, kde jednotlivé
třídy mohou reprezentovat všechna racionální čísla.
\section{Druhá přednáška}

\subsection{Algebraické struktury}
Algebraická struktura je množina, na které máme jednu, nebo více operací
a tyto operace mají nějaké vlastnosti. Obecně $(G, *)$ je algebraická
struktura na množině G s operací *. Algebraických struktur je mnoho, nas bude
zajímat převážně Grupa a Pole. Pokud bychom z následujicí definice grupy vypustili
všechny 3 podmínky, jednalo by se o tzv. Grupoid (také označován jako Magma). Při splnění první
podmínky tedy Magma a 1. podmínka, dostáváme tzv Pologrupu. Následně Pologrupou a
splněním podmínky číslo 2 dostáváme Monoid.

\subsubsection{Grupa}
\begin{definition}[Grupa]
    Grupa $(G, *)$ je algebraická struktura s jednou operací $*: G \times G \rightarrow G$,
    kde operace $*$ splňuje následujicíc vlastnosti:
    \begin{enumerate}
        \item $a * (b * c) = (a * b) * c \; \forall a, b, c \in G$ \hfill Asociativita
        \item $\exists e \in G: e * a = a * e = a \; \forall a \in G$ \hfill Neutrální prvek
        \item $\forall a \in G, \exists a^{-1} \in G: a * a^{-1} = a^{-1} * a = e$ \hfill
        Inverzní prvky
    \end{enumerate}
\end{definition}

\begin{definition}[Komutativní \uv{Abelova} grupa]
    Pokud k požadovaným vlastnostem operace $*$ tvořící grupu přidáme ještě
    čtvrtou vlastnost:
    \begin{enumerate}[start=4]
        \item $a * b = b * a \; \forall a, b \in G$ \hfill Komutativita
    \end{enumerate}
    Dostaneme tzv. Abelovskou grupu.
\end{definition}

Jako příklady grupy můžeme uvést $(\mathbb{Z, +})$, $(\mathbb{Q} \smallsetminus \{0\}, \cdot)$
$(\mathbb{Q}, +)$, $(\mathbb{R}, +)$, $(\mathbb{R} \smallsetminus \{0\}, \cdot)$
všechny tyto příklady jsou dokonce abelovskou grupou. Zajímavé je zamyslet se nad příkladem
neabelovské grupy, kterým může být například grupa permutací (permutace s operací skládání s třemi
a více prvky). Dalším příkladem neabelovské grupy je množina čtvercových regulárních matic s operací
násobení.

\begin{theorem}
    Neutrální prvek je jediný.
\end{theorem}
\begin{proof}
    Předpokládejme, že $e_1$ a $e_2$ jsou neutrální prvky. Budeme li chtít na tyto dva neutrální
    prvky aplikovat operaci $*$ podle definice neutrálního prvku vezmeme $e_1$ jako neutrální a
    dostáváme:
    $$e_1 * e_2 = e_2$$
    Zároveň ale můžeme podle definice neutrálního prvku vzít $e_2$ jako neutrální a v tom případě
    dostáváme:
    $$e_1 * e_2 = e_1$$
    Z toho vyplývá, že $e_1$ a $e_2$ jsou tentýž prvek a nemůže tedy nikdy exisovat více než
    jeden neutrální prvek.
\end{proof}

\subsubsection{Pole}
\begin{definition}[Pole]
    Pole $(F, +, \cdot)$ je algebraická struktura se dvěma operacemi, kde množina $F$ má alespoň
    dva prvky, operace $+$ splňuje následujicí vlastnosti\footnote{Všiměte si, že jsou velmi podobné
    požadovaným vlastnostem na operaci $*$ z definice Abelovy grupy.}:
    \begin{enumerate}
        \item $a + (b + c) = (a + b) + c \; \forall a, b, c \in F$ \hfill Asociativita
        \item $\exists 0_f \in F: 0_f + a = a + 0_f = a \; \forall a \in F$ \hfill Neutrální prvek
        \item $\forall a \in F, \exists -a \in F: a + (-a) = -a + a = 0_f$ \hfill Inverzní prvky
        \item $a + b = b + a \; \forall a, b \in F$ \hfill Komutativita
    \end{enumerate}
    a zároveň operace $\cdot$ splňuje:
    \begin{enumerate}
        \item $a \cdot (b \cdot c) = (a \cdot b) \cdot c \; \forall a, b, c \in F$ \hfill Asociativita
        \item $\exists 1_f \in F: 1_f \cdot a = a \cdot 1_f = a \; \forall a \in F$ \hfill Neutrální prvek
        \item $\forall a \in F \smallsetminus \{0_f\}, \exists a^{-1} \in F: a \cdot
        a^{-1} = a^{-1} \cdot a = 1_f$ \hfill Inverzní prvky
        \item $a \cdot (b + c) = a \cdot b + a \cdot c \wedge (b + c) \cdot a = b \cdot a + c \cdot a
        \; \forall a, b,c \in F$ \hfill Distributivita

    \end{enumerate}
    \label{def:field}
\end{definition}

\begin{definition}[Komutativní pole]
    Pokud se jedná o pole a navíc je operace $\cdot$ komutativní, jedná se o komutativní pole:
    \begin{enumerate}[start=5]
        \item $a \cdot b = b \cdot a \; \forall a, b \in F$ \hfill Komutativita
    \end{enumerate}
\end{definition}

Zatím jediným příkladem pole, který z přednášek známe je $(\mathbb{Q}, +, \cdot)$.\footnote{Dalším
příkladem by mohlo být pole racionálních funkcí $\mathbb{Z}(X)$, které bylo později velmi okrajově
zmíněno na přednášce.}

\begin{definition}[Uspořádané pole]
    Řekneme, že pole $F$ je uspořádané, jestliže v něm existuje $P \subseteq F$ tak, že
    je li $x, y \in P$ platí $x + y \in P \wedge x \cdot y \in P$ a dále $\forall x \in F$
    platí, že splňuje právě jednu z následujicích podmínek:
    \begin{enumerate}
        \item $x \in P$
        \item $-x \in P$
        \item $x = 0_F$
    \end{enumerate}
    \label{def:ordered_field}
\end{definition}

Jinak řečeno, uspořádané pole bude takové, ve kterém je možné nějakým způsobem vybrat \uv{kladnou}
pod\-mno\-ži\-nu. Příklady uspořádaných polí: $\mathbb{Q}, \mathbb{R}, \mathbb{Z}(x)$

Mejme uspořádané pole dle definice \ref{def:ordered_field}, potom
zavedeme relaci $<$ následovně:
$$a < b, \text{jestliže}\,b - a \in P$$

Taková relace je ostré uspořádání\footnote{To znamená, že je tato relace ireflexivní
a tranzitivní}.

\begin{definition}[Husté pole]
    Řekneme, že pole F je husté, jestliže $\forall a, b \in F, a < b$ existuje $c \in F$
    takové, že $a < c < b$.
\end{definition}

Příklady hustého pole: $\mathbb{Q}, \mathbb{R}$.

\begin{definition}[Archimédovské pole]
    Řekneme, že uspořádané pole $F$ je archimédovské, jestliže:
    $$\forall x, y \in P\,\exists n\in\mathbb{N}, n \cdot x > y $$
\end{definition}

Příklady archimédovských polí: $\mathbb{Q}, \mathbb{R}$

\subsubsection{Konečná pole}
\begin{definition}
    Konečné pole je pole $(F, +, \cdot)$, kde množina $F$ má konečný počet prvků.
\end{definition}
\begin{theorem}[Existence konečného pole]
    \label{the:field}
    Konečné pole $(F, +, \cdot)$ existuje právě tehdy, když $|F| = p^k$, kde $p$ je
    prvočíslo a $k \in \mathbb{N}$. Toto konečné pole je zároveň jediné.
\end{theorem}

Z věty \ref{the:field} vyplývá, že existují konečná pole se dvěma prvky, třemi prvky, se
čtyřmi prvky, s pěti prvky, ale ne se šesti, protože 6 není ani prvočíslo, ani mocnina prvočísla.

Konečná pole budeme značit zdvojeným fontem a počtem prvků v dolním indexu např. $\mathbb{F}_{11}$

Konečná pole si rozdělíme na dva případy a to na prvočíselná pole a na neprvočíselná pole.

Abychom porozuměli konečným polím a mohli s nimi pracovat, potřebujeme vědět,
jak na nich fungují operace $+$ a $\cdot$.

\subsubsection*{Prvočíselná pole}
\begin{definition}[Prvočíselné pole]
    Prvočíselné pole je konečné pole $(F, +, \cdot)$, kde $|F| = p$, p je prvočíslo. Tedy
    všechny případy, kdy pro $k$ z věty \ref{the:field} platí že $k=1$.
\end{definition}

Například v prvočíselném poli $\mathbb{F}_{2}$
máme 2 prvky a tyto prvky můžeme označit jak chceme, pro praktické počítání je však nejlepší
označit tyto prvky čísly, v tomto případě od $0$ do $1$, kde $0$ bude hrát roli hodnoty nula a
$1$ roli hodnoty jedna, tak jak potřebujeme.

\begin{example}[Prvočíselné pole $\mathbb{F}_{2}$]
    $$\mathbb{F}_2 = \{0, 1\}$$
    \begin{table}[h]
        \centering
        \begin{tabular}{|c|c|l|l|l|c|c|l|}
        \hline
        $+$ & $0$ & $1$ &  &  & $\cdot$ & $0$ & $1$ \\ \hline
        $0$ & $0$ & $1$ &  &  & $0$     & $0$ & $0$ \\ \hline
        $1$ & $1$ & $0$ &  &  & $1$     & $0$ & $1$ \\ \hline
        \end{tabular}
        \caption{Operace $+$ a $\cdot$ nad $\mathbb{F}_{2}$}
        \label{tab:F2}
    \end{table}

    Můžeme si všimnout, že u obou operací v tomto případě vlastně počítáme modulo 2,
    tedy modulo počet prvků pole, tato vlastnost platí obecně u prvočíselných polí.
\end{example}


\begin{example}[Prvočíselné pole $\mathbb{F}_{5}$]
    $$\mathbb{F}_5 = \{0, 1, 2, 3, 4\}$$
\begin{table}[h]
    \centering
    \begin{tabular}{|c|c|c|c|c|c|c|c|c|c|c|c|c|c|}
    \hline
    $+$ &
      $0$ &
      $1$ &
      $2$ &
      $3$ &
      $4$ &
       &
       &
      $\cdot$ &
      $0$ &
      $1$ &
      $2$ &
      $3$ &
      $4$ \\ \hline
    $0$ &
      $0$ &
      $1$ &
      $2$ &
      $3$ &
      $4$ &
       &
       &
      $0$ &
      $0$ &
      $0$ &
      $0$ &
      $0$ &
      $0$ \\ \hline
    $1$ &
      $1$ &
      \cellcolor[HTML]{FFFFFF}$2$ &
      \cellcolor[HTML]{FFFFFF}$3$ &
      \cellcolor[HTML]{FFFFFF}$4$ &
      \cellcolor[HTML]{FFFFFF}$0$ &
       &
       &
      $1$ &
      $0$ &
      \cellcolor[HTML]{34FF34}$1$ &
      \cellcolor[HTML]{34FF34}$2$ &
      \cellcolor[HTML]{34FF34}$3$ &
      \cellcolor[HTML]{34FF34}$4$ \\ \hline
    $2$ &
      $2$ &
      \cellcolor[HTML]{FFFFFF}$3$ &
      \cellcolor[HTML]{FFFFFF}$4$ &
      \cellcolor[HTML]{FFFFFF}$0$ &
      \cellcolor[HTML]{FFFFFF}$1$ &
       &
       &
      $2$ &
      $0$ &
      \cellcolor[HTML]{34FF34}$2$ &
      \cellcolor[HTML]{34FF34}$4$ &
      \cellcolor[HTML]{34FF34}$1$ &
      \cellcolor[HTML]{34FF34}$3$ \\ \hline
    $3$ &
      $3$ &
      \cellcolor[HTML]{FFFFFF}$4$ &
      \cellcolor[HTML]{FFFFFF}$0$ &
      \cellcolor[HTML]{FFFFFF}$1$ &
      \cellcolor[HTML]{FFFFFF}$2$ &
       &
       &
      $3$ &
      $0$ &
      \cellcolor[HTML]{34FF34}$3$ &
      \cellcolor[HTML]{34FF34}$1$ &
      \cellcolor[HTML]{34FF34}$4$ &
      \cellcolor[HTML]{34FF34}$2$ \\ \hline
    $4$ &
      $4$ &
      \cellcolor[HTML]{FFFFFF}$0$ &
      \cellcolor[HTML]{FFFFFF}$1$ &
      \cellcolor[HTML]{FFFFFF}$2$ &
      \cellcolor[HTML]{FFFFFF}$3$ &
       &
       &
      $4$ &
      $0$ &
      \cellcolor[HTML]{34FF34}$4$ &
      \cellcolor[HTML]{34FF34}$3$ &
      \cellcolor[HTML]{34FF34}$2$ &
      \cellcolor[HTML]{34FF34}$1$ \\ \hline
    \end{tabular}
    \caption{Operace $+$ a $\cdot$ nad $\mathbb{F}_{5}$}
    \label{tab:F5}
    \end{table}

    Z definice pole\ref{def:field} vyplývá, že $(\mathbb{F}_5, +)$ musí tvořit Abelovskou grupu.
    V Abelovské grupě platí, že při rozepsání operace do tabulky je v každém sloupci
    a v každém řádku každý prvek obsažen právě jednou\footnote{V počátcích definic
    teorie grup se tato vlastnost používala pro definici grupy.}. Což si můžeme všimnout
    že zde platí.

    U operace $\cdot$ si můžeme všimnout, že bez prvního sloupce a bez prvního řádku
    (zeleně označená část) operace $\cdot$ tvoří grupu. Tato vlastnost u pole a jeho
    operace $\cdot$ platí vždy.
\end{example}

\subsubsection*{Neprvočíselná pole}
\begin{definition}[Neprvočíselné pole]
    Neprvočíselné pole je konečné pole $(F, +, \cdot)$, kde $|F| = p^k$, $p$ je prvočíslo a
    zároveň $k > 1, k \in \mathbb{N}$. Tedy všechny případy, kdy pro $k$ z věty \ref{the:field}
    platí, že $k>1$.
\end{definition}

V případě neprvočíselných polí nebude fungování operací tak zřejmé jako tomu bylo u
prvočíselných polí. Použitím stejného triku
jako u prvočíselných polí, tedy použití běžných operací $+$ a $\cdot$ modulo počet prvků,
totiž nejsme schopni vytvořit pole. Problém je v takovém případě operace $\cdot$, kdy
pouze s přidáním modula nebude splňovat požadované vlastnosti z definice pole\ref{def:field}.

Opět platí, že prvky pole můžeme označit jak chceme, ale je dobré, udělat to tak, aby se nám
s nimi vhodně pracovalo. V případě neprvočíselných polí je pro jejich odvození vhodné
označit si prvky jako polynomy v proměnné $t$, kde koeficienty jsou z $\mathbb{F}_p$ až do
stupně $k - 1$, kde $n = p^k$ pro $\mathbb{F}_n$.

\begin{example}[Definice pro $\mathbb{F}_4$]
    $$4 = 2^2,\; p = 2,\; k = 2$$
    Polynomy v tomto případě tedy budou:
    \begin{table}[h]
        \centering
        \begin{tabular}{|c|c|c|c|c|}
        \hline
        Polynomy          & $0$ & $1$ & $t$ & $t + 1$ \\ \hline
        Pomyslná hodnota & $0$ & $1$ & $2$ & $3$      \\ \hline
        \end{tabular}
        \caption{Vyjádření polynomů pro $\mathbb{F}_{4}$}
        \label{tab:F4_pol}
        \end{table}
\end{example}

Pro vytvoření aditivní operace stačí sčítat polynomy v každém stupni modulo $p$.

\begin{example}[Tvorba aditivní operace pro $\mathbb{F}_4$]
Budeme sčítat polynomy v každém stupni modulo $p$
\[
2 + 3 = t + (t + 1)
=
\begin{array}{rr}
    t & + 0\\
    t & + 1\\ \hline
    0 & + 1\\
      &
\end{array}
= 1
\]

\[
1 + 1 = 1 + 1 = t
=
\begin{array}{rr}
    0t & + 1\\
    0t & + 1\\ \hline
    0t & + 0\\
       &
\end{array}
= 0
\]
\[
    1 + 2 = 1 + t
=
\begin{array}{rr}
    0t & + 1\\
    t  & + 0\\ \hline
    t  & + 1\\
       &
\end{array}
= 3
\]
Stejným postupem pro ostatní hodnoty (některé jdou rovnou doplnit díky vlasnostem operace $+$)
dostaneme kompletní tabulku definujicí aditivní operaci $+$.

\begin{table}[h]
    \centering
    \begin{tabular}{|l|l|l|l|l|}
    \hline
    \multicolumn{1}{|c|}{$+$} & \multicolumn{1}{c|}{$0$} & \multicolumn{1}{c|}{$1$} & \multicolumn{1}{c|}{$2$} & \multicolumn{1}{c|}{$3$} \\ \hline
    \multicolumn{1}{|c|}{$0$} & \multicolumn{1}{c|}{$0$} & \multicolumn{1}{c|}{$1$} & \multicolumn{1}{c|}{$2$} & \multicolumn{1}{c|}{$3$} \\ \hline
    $1$                       & $1$                      & $0$                      & $3$                      & $2$                      \\ \hline
    $2$                       & $2$                      & $3$                      & $0$                      & $1$                      \\ \hline
    $3$                       & $3$                      & $2$                      & $1$                      & $0$                      \\ \hline
    \end{tabular}
    \caption{Aditivní operace pro $\mathbb{F}_{4}$}
    \label{tab:F4_plus}
    \end{table}

\end{example}

\begin{example}[Příklad polynomů pro $\mathbb{F}_{125}$]
    $$125 = 5^3, \; p = 5, \; k = 3$$
    Polynomy budou následující:
    $$0, \; 1, \; 2, \; 3, \; 4, \; t, \; t + 1, \; t + 2, \; t + 3, \; t + 4,
        2t + 1,\; \dots \;4t^2 + 4t + 3, \;  4t^2 + 4t + 4$$

Ukázka součtu dvou polynomů:
\[
    (4t + 2) + (t^2 + 2t + 3)
    =
    \begin{array}{rrr}
        0t^2 & + 4t & + 2\\
        t^2  & + 2t & + 3\\ \hline
        t^2  & + 1t & + 0\\
          &
    \end{array}
= t^2 + t
\]
\end{example}

Při vytváření multiplikativní operace se nám stane, že po vynásobení dvou polynomů
vznikne polynom stupně, který je větší, než $k - 1$ a tedy není mezi polynomy daného pole.
Budeme proto potřebovat tzv. redukční polynom.
\begin{definition}[Redukční polynom]
    Redukční polynom $P_{red}:$ polynom stupně $k$, který je nerozložitelný na součin
    polynomů stupně nižších (řekneme, že je ireducibilní).
\end{definition}


\begin{example}[Hledání redukčního polynomu pro $\mathbb{F}_4$]

Všechny polynomy stupně $k=2$:
    \begin{itemize}
        \item $t^2$ lze rozložit na $t \cdot t$
        \item $t^2 + 1$ lze rozložit na $(t + 1) \cdot (t + 1)$
        \item $t^2 + t$ lze rozložit na $t \cdot (t + 1)$
        \item $t^2 + t + 1$ nelze rozložit
    \end{itemize}

\[
    (t + 1) \cdot (t + 1) =
    \begin{array}{rrr}
        &t&+1 \\
        &t&+1 \\ \hline
        &t&+1 \\
        t^2&+t& \\ \hline
        t^2&+0t&+1
    \end{array}
= t^2 + 1
\]
\end{example}

Tvorba multiplikativní operace: po vynásobení dvou prvků z $\mathbb{F}_{p^k}$ vyjádřených pomocí
polynomů odečítáme (je-li třeba) $t^h \cdot P_{red}$ tak dlouho, až je výsledek stupně nejvýše
$k - 1$.

\begin{example}[Aplikace multiplikativní operace v $\mathbb{F}_4$ a využití $P_{red}$]
$$t \cdot (t + 1) = t^2 + t$$
Polynom $t^2 + t$ má ale příliš vysoký stupeň (vyšší, než $k - 1$). Začneme proto s odečítáním
redukčního polynomu\footnote{Můžeme odečítat i jeho $t^h$ násobky, ale v tomto případě stačí
redukční polynom sám o sobě.}, který je v tomto případě $t^2 + t + 1$.
\[
    (t^2 + t) - (t^2 + t + 1)
    =
    \begin{array}{rrr}
        t^2&+t&+0 \\
       -(t^2&+t&+1) \\ \hline
        0t^2&+0t&+1 \\
        &&
    \end{array}
    = 1
\]
\end{example}

\begin{example}[Tvorba tabulky multiplikativní operace v $\mathbb{F}_4$]
    Hodnoty pro $0$ a $1$ jsou jasné. V předchozím příkladu jsme spočítali,
    že $3 \cdot 2 = 1$, díky čemuž zároveň víme že, $2 \cdot 3 = 1$. Ostatní
    hodnoty jsme již schopni doplnit díky požadovaným vlastnostem operace $\cdot$.
    Ale pojďme ověřit $2 \cdot 2$.

    $$2 \cdot 2 = t \cdot t = t^2$$
    Stupeň polynomu je větší, než $k - 1$. Odečteme $T_{red}$.
    \[
        t^2 - (t^2 + t + 1)
        =
        \begin{array}{rrr}
            t^2&+0t&+0 \\
           -(t^2&+t&+1) \\ \hline
            0t^2&+t&+1 \\
            &&
        \end{array}
        = t + 1 = 3
    \]

    \begin{table}[h]
        \centering
        \begin{tabular}{|c|c|c|c|c|}
        \hline
        $\cdot$ & $0$ & $1$ & $2$ & $3$ \\ \hline
        $0$     & $0$ & $0$ & $0$ & $0$ \\ \hline
        $1$     & $0$ & $1$ & $2$ & $3$ \\ \hline
        $2$     & $0$ & $2$ & $3$ & $1$ \\ \hline
        $3$     & $0$ & $3$ & $1$ & $2$ \\ \hline
        \end{tabular}
        \caption{Multiplikativní operace pro $\mathbb{F}_{4}$}
        \label{tab:F4_mul}
    \end{table}
\end{example}

\subsection{Konstrukce množiny reálných čísel}
Využijeme definici reálných čísel pomocí Dedekindových řezů.

\begin{definition}[Dedekinduv řez]
    Dedekindův řez $D$ je podmnožina racionálních čísel $D \subseteq \mathbb{Q}$, která splňuje:
    \begin{enumerate}
        \item $x \in D \Rightarrow \exists y > x, y \in D$ \hfill Neexistence největšího prvku
        \item $x \in D, y < x \Rightarrow y \in D$ \hfill
    \end{enumerate}
\end{definition}

Příklady Dedekindových řezů:
\begin{itemize}
    \item $\mathbb{Q}$ tento řez označme $\infty$
    \item $\emptyset$ tento řez označme $- \infty$
    \item $\mathbb{Q}^{-}$
    \item $\{x \in \mathbb{Q};\,x < 7\}$
    \item $\{x \in \mathbb{Q};\, x \cdot x < 2 \vee x < 0\}$
\end{itemize}

Budeme li uvažovat všechny dedekindovy řezy, dostaneme množinu rozšířených reálných
čísel, kterou budeme označovat $\overline{\mathbb{R}}$.

Potom $$\mathbb{R} = \overline{\mathbb{R}} \smallsetminus \{-\infty, \infty\}$$
Kde $\mathbb{R}$ označuje množinu reálných čísel.

\begin{definition}[Součet Dedekindových řezů]
    $$D + E = \{x + y; x\in D, y \in E\}$$
\end{definition}

\begin{definition}[Nezáporný dedekindův řez]
    Řekneme, že dedekindův řez $D$ je nezáporný právě tehdy, když:
    $$D \supseteq \mathbb{Q}^{-}$$
\end{definition}

\begin{definition}[Součin Dedekindových řezů]
    Předpokládáme, že řezy $D$ a $E$ jsou nezáporné.
    $$D \cdot E = \{x \cdot y; \forall x, y \geq 0, x \in D, y \in E\} \cup \{z; z < 0, z \in \mathbb{Q}\}$$
    Pokud je jeden z řezů záporný a druhý nezáporný, potom musíme definovat opačný řez,
    k zápornému řezu vyrobit řez opačný, použijeme násobení nezáporných řezů a z výsledku opět vyrobíme
    řez opačný.

    Pokud budou oba řezy záporné, z obou řezů vezmu opačné řezy, použiji násobení nezáporných řezů
    a dostanu korektní výsledek. \footnote{Násobení dedekindových řezů bylo na přednášce definováno
    pouze takto částečně.}
\end{definition}

\subsubsection*{Komplexní čísla}
Uvažujme $\mathbb{R} \times \mathbb{R} = \mathbb{R}^2$. V $\mathbb{R}^2$ není
definována multiplukativní operace $(a, b) \cdot (c, d)$. Pokud v $\mathbb{R}^2$
multiplikativní operaci definujeme takovým způsobem, aby splňovala
vlastnosti na multiplikativní operaci z definice pole\ref{def:field}, dostáváme:

$$(a, b) \cdot (c, d) = (ac - bd, ad + bc)$$

Což je totéž jako
$$(a + bi) \cdot (c + di) = ac + (b + d)i^2 + a \cdot di + bi \cdot c = ac - bd + (ad + bc)i, \; i^2 = -1$$

Přidáním této operace dostaneme množinu komplexních čísel $\mathbb{C}$, která opět tvoří
strukturu pole.

Byly snahy tento postup zobecnit. Pro $\mathbb{R}^3$ avšak vhodná multiplikativní operace, která by
vyhovovala požadavkům z definice pole\ref{def:field} neexistuje.

Pro $\mathbb{R}^4$ už multiplikativní operaci splňujicí požadované vlastnosti vytvořit
lze, tím dostáváme tzv. kvaterniony, značíme je $\mathbb{H}$.
Opět máme \uv{pomůcky} a pravidla pro jejich násobení.
Kvaterniony zapisujeme ve tvaru:
$$a + bi + cj + dk$$
A pravidla pro jejich násobení jsou:
\begin{enumerate}
    \item $i^2 = j^2 = k^2 = -1$
    \item $ij = -ji = k$
\end{enumerate}
U kvaternionů však máme jednu změnu, nejedná se o komutativní pole (je to zřejmé z druhého pravidla)
a jsou tedy prvním příkladem nekomutativního pole se kterým jsme se v přednáškách zatím setkali.

\subsection{Mohutnosti nekonečných množin}
Kardinalita nejvšednější nekonečné množiny, přirozených čísel, je definována jako \uv{alef 0}
$$|\mathbb{N}| = \aleph_0$$
Jakákoliv jiná nekonečná množina bude mít stejnou kardinalitu, pokud existuje bijekce
mezi touto nekonečnou množinou a množinou přirozených čísel.

\begin{table}[h]
    \centering
    \begin{tabular}{cccccccccccc}
    $\mathbb{N}$   & $1$ & $2$ & $3$  & $4$ & $5$  & $6$  & $7$  & $8$  & $9$  & $10$ & $\dots$ \\
    $\mathbb{N}_0$ & $0$ & $1$ & $2$  & $3$ & $4$  & $5$  & $6$  & $7$  & $8$  & $9$  & $\dots$ \\
    $2\mathbb{N}+1$   & $1$ & $3$ & $5$  & $7$ & $9$  & $11$ & $13$ & $15$ & $17$ & $19$ & $\dots$ \\
    $\mathbb{Z}$  & $0$ & $1$ & $-1$ & $2$ & $-2$ & $3$  & $-3$ & $4$  & $-4$ & $5$  & $\dots$ \\
    $\mathbb{Q}$ &
      $\frac{0}{1}$ &
      $\frac{-1}{1}$ &
      $\frac{-2}{1}$ &
      $\frac{-1}{2}$ &
      $\frac{1}{2}$ &
      $\frac{2}{1}$ &
      $\frac{-3}{1}$ &
      $\frac{-1}{3}$ &
      $\frac{1}{3}$ &
      $\frac{3}{1}$ &
      $\dots$
    \end{tabular}
    \caption{Ukázka některých bijekcí s přirozenými čísly}
    \label{tab:Naturlas_bijection}
\end{table}

Z bijekcí naznačených v tabulce \ref{tab:Naturlas_bijection} vyplývá:
$$|\mathbb{N}| = |\mathbb{N}_0| = |2\mathbb{N}+1| = |\mathbb{Z}| = |\mathbb{Q}| = \aleph_0$$

\subsubsection*{Mohutnost množiny reálných čísel}
Mohutnost množiny reaálných čísel je větší, než mohutnost množiny celých čísel.
$$|\mathbb{R}| > |\mathbb{N}|$$
\begin{proof}
    Neexistence bijekce mezi $\mathbb{R}$ a $\mathbb{N}$

    Předpokládejme, že bijekce mezi $\mathbb{R}$ a $\mathbb{N}$ existuje.
    Vezměme reálný interval $\langle0, 1)$ a předpokládejme, že jeho prvky lze seřadit\footnote{Tento
    předpoklad vychází z předpokladu, že existuje bijekce s $\mathbb{N}$.}.

    Za předpokladu, že jsme schopni hodnoty tohoto intervalu seřadit, jsme schopni
    je všechny reprezentovat nekonečnou tabulkou \ref{tab:diag_real}.


\begin{table}[]
    \centering
    \begin{tabular}{cccccl}
    $a_1 = $ & $0,$     & \cellcolor[HTML]{3166FF}$a_{11}$ & $a_{12}$                         & $a_{13}$                         & $\dots$  \\
    $a_2 = $ & $0,$     & $a_{21}$                         & \cellcolor[HTML]{3166FF}$a_{22}$ & $a_{23}$                         & $\dots$  \\
    $a_3 = $ & $0,$     & $a_{31}$                         & $a_{32}$                         & \cellcolor[HTML]{3166FF}$a_{33}$ & $\dots$  \\
    $\vdots$ & $\vdots$ & $\vdots$                         & $\vdots$                         & $\vdots$                         & $\ddots$
    \end{tabular}
    \caption{Seřazení hodnot reálného intervalu $\langle 0, 1)$}
    \label{tab:diag_real}
\end{table}

    Teď vytvoříme číslo $b = 0,b_1 b_2 b_3 \ldots$, kde každou číslici $b_i$ určíme následovně:
    \[
    b_i =
    \left\{
    \begin{array}{ll}
        1 & \text{pokud} \; a_{ii} \neq 1\\
        2 & \text{pokud} \; a_{ii} = 1   \\
    \end{array}
    \right.
\]

    Tím jsme ale zkonstruovali reálné číslo $b$, které se liší\footnote{A to alespoň v
    jedné číslici na diagonále (zobrazeno modře).} od každého čísla v tabulce
    \ref{tab:diag_real}.

    Z našich předpokladů však vycházelo, že v tabulce musí být obsažena všechna čísla z daného
    intervalu. Dostáváme tedy spor a z toho vychází, že naše předpoklady nebyly správné a
    neexistuje bijekce mezi $\mathbb{N}$ a reálným intervalem $\langle 0, 1)$. Tím pádem nemůže
    existovat bijekce ani mezi $\mathbb{R}$ a $\mathbb{N}$.

    Z důkazu nám zároveň vychází, že $|\mathbb{R}| > |\mathbb{N}|$.
\end{proof}

Kardinalitu reálných čísel budeme značit $c$
$$|\mathbb{R}| = c > \aleph_0$$

\begin{definition}[Kardinalita potenčních množin přirozených čísel]
    Značíme pomocí $\aleph_i$
    $$|P(\mathbb{N})| = \aleph_1$$
    $$|P(P(\mathbb{N}))| = \aleph_2$$
    $$|P(P(P(\mathbb{N})))| = \aleph_3$$
    $$\vdots$$
    Kde
    $$\aleph_0 < \aleph_1 < \aleph_2 < \aleph_3 \ldots$$
\end{definition}

\begin{proof}
    Neexistence bijekce mezi $\mathbb{N}$ a $P(\mathbb{N})$

    Předpokládejme, že $f: \mathbb{N} \rightarrow P(\mathbb{N})$ je bijekce.

    Nyní uvažujme množinu
    $$D = \{n \in \mathbb{N}; n \notin f(n)\}$$
    $D$ je nějaká podmnožina všech přirozených čísel $a$, kde bijekce $f$ zobrazí $a$ na
    podmnožinu, která číslo $a$ neobsahuje.

    Vzhledem k tomu, že $D \subseteq \mathbb{N}$, musí platit $D \in P(\mathbb{N})$, pak
    $$\exists m \in \mathbb{N}: f(m) = D$$
    Potom ale $$m \in D \Leftrightarrow m \notin D$$
    Čímž se dostáváme ke sporu a bijekce $f$ jejíž existenci jsme předpokládali neexistuje.
\end{proof}

Není jednoznačné, zda $\aleph_1 = c$ \footnote{Jedná se o nezávislý axiom.}.

\subsubsection*{Mohutnost množiny komplexních čísel}
$$|\mathbb{C}| = |\mathbb{R}^2| = |\mathbb{R}| = c$$
Což ovšem znamená, že musíme být schopni najít bijekci mezi $\mathbb{R}$ a $\mathbb{R}^2$.

Toho dosáhneme následovně, každé reálné číslo zobrazíme na uspořádanou dvojici takto:
$$0,3451239956\ldots \rightarrow (0,35295\ldots ;0,41396\ldots)$$

Tímto způsobem jsme schopni obecně najít bijekci mezi $\mathbb{R}$ a $\mathbb{R}^n$.

\section{Třetí přednáška}

\subsection{Vektorový prostor}
\begin{definition}[Vektorový prostor]

    Vektorový prostor je především Abelovská grupa
    $$(\mathcal{V}, +)$$
    Kromě operace + budeme mít další operaci vztaženou k nějakému poli $(F, +, \cdot)$
    a to operaci
    $$\cdot: F \times \mathcal{V} \rightarrow \mathcal{V}$$
    této operaci budeme říkat násobení skalárem a bude mít následujicí vlastnosti:
    \begin{enumerate}
        \item $a \cdot (\vec{u} + \vec{v}) = a \cdot \vec{u} + a \cdot \vec{v}, \forall
            a \in F \; \forall \vec{u}, \vec{v} \in \mathcal{V}$
        \item $(a + b) \cdot \vec{u} = a \cdot \vec{u} + b \cdot \vec{u},
            \forall a,b \in F \; \forall \vec{u} \in V$
        \item $(a \cdot b ) \cdot \vec{u} = a \cdot (b \cdot \vec{u}),
            \forall a,b \in F \; \forall \vec{u} \in V$
        \item $1 \cdot \vec{u} = \vec{u}, \forall \vec{u} \in \mathcal{V}$
    \end{enumerate}
    Tedy pokud máme nějaké pole $F$, nějakou abelovskou grupu $\mathcal{V}$ a definovaný
    skalární součin $\cdot$ s uvedenými vlastnostmi. Potom řekneme, že $\mathcal{V}$ je
    vektorovým prostorem nad polem $F$.

    Prvkům $\mathcal{V}$ pak budeme říkat vektory. Prvkům $F$ často říkáme skaláry,
    nebo čísla.
    \label{def:vector_space}
\end{definition}
\begin{example}[Příklad vektorového prostoru]
    $$F = \mathbb{R}, \mathcal{V} = \mathbb{R}^3$$
    $$\vec{u} + \vec{v} = (u_1 + v_1, u_2 + v_2, u_3 + v_3)
    \forall \vec{u}, \vec{v} \in \mathcal{V}$$
    Je $(\mathcal{V}, +)$ abelovská grupa?
    \begin{enumerate}[start=0]
        \item Operace $+$ je zjevně uzavřená na množině $\mathbb{R}^3$.
        \item Asociativita vychází z asociativity sčítání v $\mathbb{R}$.
        \item Neutrálním prvkem je: $\vec{o} = (0, 0, 0)$.
        \item Inverzní prvek k prvku $u$ dostaneme jako: $-\vec{u} = (-u_1, -u_2, -u_3)$.
        \item Komutativita je v tomto případě také jasná.
    \end{enumerate}
    $(\mathcal{V}, +)$ je tedy abelovská grupa\footnote{Otázka k zamyšlení je, zda by bylo
    možné operaci $+$ zavést nějak jinak a stále dodržet všechny požadované vlastnosti.}.

    operaci násobení skalárem $\cdot$ zavedeme takto:
    $$a \cdot (u_1, u_2, u_3) = (a \cdot u_1, a \cdot u_2, a \cdot u_3)$$
    Je zřejmé, že tato operace splňuje všechny požadované vlastnosti z definice
    vektorového prostoru \ref{def:vector_space}. Pojďme ověřit například třetí vlastnost:
    \begin{proof} Platnost třetí vlasnosti:
        $$LS: (a \cdot b) \cdot (u_1, u_2, u_3) =
        (a \cdot b \cdot u_1, a \cdot b \cdot u_2, a \cdot b \cdot u_3)$$
        $$PS: a\cdot (b \cdot (u_1, u_2, u_3)) = a \cdot (b \cdot u_1, b \cdot u_2, b \cdot u_3) =
        (a \cdot b \cdot u_1, a \cdot b \cdot u_2, a \cdot b \cdot u_3)$$
        $$LS = PS$$
    \end{proof}
    \label{ex:vec_space}
\end{example}

Ověření platnosti požadovaných vlastností bylo v případě \ref{ex:vec_space} triviální.
Podobně jednoduchý postup by šel použít i pro případy, kdy obecně
$\mathcal{V} = \mathbb{R}^n, \forall n \in \mathbb{N}$. Existují však příklady, ke kterým
se dostaneme později, kdy ověření platnosti není tak jednoduché.

\begin{definition}[Triviální vektorový prostor]
    Uvažujme libovolné pole $F$ a abelovskou grupu
    $$(V, +), \mathcal{V} = \{\vec{o}\}$$
    Což je zároveň minimální případ grupy, protože ta z definice vždy musí
    obsahovat minimálně jeden prvek, kterým je neutrální prvek.

    A operaci násobení skalárem zavedeme takto:
    $$a \cdot \vec{o} = \vec{o}, a \in F, \vec{o} \in \mathcal{V}$$
    Opět budou splněny všechny požadované vlastnosti, jedná se tedy o vektorový prostor
    a tento vektorový prostor budeme označovat jako triviální vektorový prostor.
\end{definition}

Dalším jednoduchý příklad vektorového prostorů:
libovolné pole $F$, $\mathcal{V} = F^n, n \in \mathbb{N}$

\begin{example}[Je $\mathbb{R}$ vektorový prostor nad $\mathbb{Q}$?]
    $$F = \mathbb{Q}, \mathcal{V} = \mathbb{R}$$
    Víme, že $\mathbb{R}$ s běžnou operací $+$ tvoří Abelovskou grupu.
    Násobení skalárem zavedeme jako běžné násobení.
    To že platí vlastnosti z definice vektorového prostoru\ref{def:vector_space}
    plyne z vlastností bežné operace násobení $\cdot$ na množině $\mathbb{R}$.

    Takže ano, v tomto případě se jedná o vektorový prostor.
\end{example}

Další příklady vektorových prostorů:
\begin{itemize}
    \item $F = \mathbb{R}, \mathcal{V} = C^0\langle a, b \rangle$
        \footnote{$C^n\langle a,b \rangle$ značí množinu funkcí na reálném intervalu
        $\langle a,b \rangle$, které jsou spojité až do n-té derivace, $n \in \mathbb{N}$.}
    \item $F = \mathbb{R}, \mathcal{V} = C^1\langle a, b \rangle$
    \item $F = \mathbb{R}, \mathcal{V}=$ množina matic o velikosti $n \times n$
    \item $F = \mathbb{Q}, \mathcal{V} = \mathbb{R}$
    \item $F = \mathbb{R}, \mathcal{V} = \mathbb{C}$
    \item A dokonce i $F = \mathbb{C}, \mathcal{V} = \mathbb{R}$, v tomto případě však bude
        vytvoření vhodných operací netriviální.
\end{itemize}

\begin{definition}[Vektorový podprostor]
    Nechť $\mathcal{V}$ je vektorový prostor
    $$\mathcal{W} \subseteq \mathcal{V}$$
    A zároveň platí tyto vlastnosti:
    \begin{enumerate}
        \item $\vec{u}, \vec{v} \in \mathcal{W} \Rightarrow \vec{u} + \vec{v} \in \mathcal{W}$
        \item $\vec{u} \in \mathcal{W}, a \in F \Rightarrow  a \cdot \vec{u} \in \mathcal{W}$
    \end{enumerate}

    Pokud je $\mathcal{W}$ podmnožina $\mathcal{V}$ a zároveň splňuje dvě výše uvedené vlastnosti,
    potom řekneme, že $\mathcal{W}$ je vektorovým podprostorem vektorového prostu $\mathcal{V}$.

    Sjednocení dvou vektorových podprostorů obecně není vektorový podprostor.
    Průnik podprostorů je podprostor.
\end{definition}

\begin{example}[Ověření vektorového podprostoru]
    Mějme
    $$F = \mathbb{R}, \mathcal{V} = \mathbb{R}^3$$
    $$\mathcal{W}_1 = \{(a, 2\cdot a, 1), a \in \mathbb{R}\}$$

    Je $\mathcal{W}_1$ vektorový podprostor $\mathcal{V}$?
    \begin{enumerate}[start=0]
        \item Je zřejmé, že $\mathcal{W}_1 \subseteq \mathbb{R}^3$
        \item $(a, 2 \cdot a, 1) + (b, 2 \cdot b, 1) = (a + b, 2 \cdot a + 2 \cdot b, 2)
        \notin \mathcal{W}_1$
    \end{enumerate}
    První podmínka tedy není splněna a $\mathcal{W}_1$ v tomto případě není vektorovým podprostorem
    vektorového prostoru $\mathcal{V}$.

    Nyní uvažme $\mathcal{W}_2$ definované následovně:
    $$\mathcal{W}_2 = \{(a, 2\cdot a, 0), a \in \mathbb{R}\}$$
    \begin{enumerate}[start=0]
        \item Opět je zřejmé, že $\mathcal{W}_2 \subseteq \mathbb{R}^3$
        \item $(a, 2 \cdot a, 0) + (b, 2 \cdot b, 0) = (a + b, 2 \cdot (a + b), 0) = (c, 2 \cdot c, 0) \in \mathcal{W}$
        \item $a \cdot (b, 2b, 0) = (a \cdot b, 2\cdot a \cdot b, 0) = (c, 2c, 0) \in \mathcal{W}$
    \end{enumerate}
    Všechny požadované vlastnosti platí a $\mathcal{W}$ je tedy vektorový podprostor vektorového
    prostoru $\mathcal{V}$.
\end{example}

\subsection{Matice}

Intuitivně můžeme matici definovat jako čísla, která jsou uspořádaná do obdélníkového schématu.

\begin{definition}[Matice]
    Uvažujme zobrazení $a$ definované takto:
    $$a = \{1 \ldots m\} \times \{1 \ldots n\} \rightarrow F$$

    Toto zobrazení vlastně přiřazuje dvojici indexů (sloupcový a řádkový) hodnotu,
    která se v matici nachází na daném indexu. Indexy budeme značit dolním indexem
    jako $a_{mn}$
\end{definition}

Psát matice jako zobrazení by bylo silně nepraktické, proto nadále budeme využívat
právě ono uspořádání čísel do obdélníkového schématu a označovat je velkými písmeny.

Prvky $a_{11}, a_{22}, \ldots, a_{kk}, k = min(m, n)$ budeme označovat jako hlavní
diagonálu.

Dále budeme často používat indexování pomocí $i, j$, kde
$$i = 1, \ldots, m$$
$$j = 1, \ldots, n$$

\begin{example}[Matice]

    Mějme matici A definovanou takto:
    $$ A = \begin{pmatrix}
        3 & 2 & 1 & -5 \\
        2 & \sqrt{2} & 3 & -\frac{1}{3}
        \end{pmatrix}  $$

    Potom $a(2, 2) = a_{22} = \sqrt{2}$, $a(2, 4) = a_{24} = -\sqrt{1}{3}$
\end{example}

\subsection{Speciální případy matic}
Mezi maticemi rozlišujeme celou řadu speciálních případů. Některé z nich jsou pouze
pro případ matic čtvercových, ale některé se dají definovat i obecně pro obdélníkové matice.

\begin{definition}[Horní trojúhelníková matice]
    Jestliže jsou v matici $A$ všechny prvky pod hlavní diagonálou 0,
    řekneme že matice $A$ je horní trojúhelníková.

    Řečeno formálně, v matici musí platit:
    $$i > j \Rightarrow a_{ij} = 0$$

    Podobně je definovaná dolní trojúhelníková matice, pouze se $i < j$ změni na $i > j$.
\end{definition}

\begin{definition}[Diagonální matice]
    Jestliže jsou v matici $A$ všechny prvky mimo hlavní diagonálou 0,
    řekneme že matice $A$ je diagonální.

    Řečeno formálně, v matici musí platit:
    $$i \neq j \Rightarrow a_{ij} = 0$$
\end{definition}

\begin{definition}[Nulová matice]
    Jestliže jsou v matici $A$ všechny prvky 0,
    řekneme že matice $A$ je nulová.

    Řečeno formálně, v matici musí platit:
    $$\forall i,j: a_{ij} = 0$$
\end{definition}

\begin{definition}[Jednotková matice]
    Jestliže je matice $A$ diagonální a zároveň jsou všechny hodnoty na hlavní diagonále 1,
    řekneme, že matice $A$ je jednotková.

    Řečeno formálně, v matici musí platit:
    \[
        a_{ij} = \delta_{ij}\footnote{Kroneckerovo delta} =
        \left\{
            \begin{array}{ll}
                1 & \text{ pro } i = j\\
                0 & \text{ pro } i \neq j\\
            \end{array}
        \right.
    \]

    Jednotkovou matici budeme značit $E$.
\end{definition}

\subsection{Základní operace na maticích}

Množinu všech matic o $m$ řádcích a $n$ sloupcích nad polem F
budeme označovat následovně $$Mat_{m,n}(F)$$

\begin{definition}[Součet matic]
    $$+: Mat_{m,n}(F) \times Mat_{m,n}(F) \rightarrow Mat_{m,n}(F)$$
    $$c_{ij} = a_{ij} + b_{ij}\; \forall i,j$$

    Tato operace je uzavřená, asociativní, má definovaný neutrální i inverzní prvek a navíc
    je komutativní.

    A tvoří tedy Abelovskou grupu $(Mat_{m,n}(F), +)$
    \label{def:mat_plus}
\end{definition}

\begin{definition}[Násobení matice skalárem]
    $$\cdot: F \times Mat_{m,n}(F) \rightarrow Mat_{m,n}(F)$$
    $$b_{ij} = k \cdot a_{ij}$$

    Tato operace splňuje požadavky pro operaci násobení skalárem z
    definice vektorového prostoru \ref{ex:vec_space}.

    A $Mat_{m,n}(F)$ v kombinaci s operací $+$ z definice součtu matic\ref{def:mat_plus}
    a s touto operací $\cdot$ tvoří vektorový prostor:
    $$(Mat_{m,n}(F), +, \cdot)$$
\end{definition}

\begin{definition}[Transpozice matice]
    $$Mat_{m,n}(F) \rightarrow Mat_{m,n}(F)$$
    $$B = A^T$$
    $$b_{ji} = a_{ij}$$
    $$(A^T)^T = A$$
    $$(A + B)^T = A^T + B^T$$
\end{definition}

\begin{definition}[Násobení matic]
    $$\cdot: Mat_{m,n}(F) \times Mat_{m,p}(F) \rightarrow Mat_{m,p}(F)$$
    $$c_{ij} = \sum_{k=1}^n a_{ik} \cdot b_{kj}$$

    Tato operace nemusí být vždy definována (musí správně sedět dimenze).
    Je to operace asociativní. Ale není komutativní.

    Omezíme li se na čtvercové regulární matice řádu $n$, dostaneme v kombinaci s touto operací
    strukturu grupy.
\end{definition}


\section{Čtvrtá přednáška}
\subsection{Vektorové prostory}

\begin{definition}[Lineární obal]
    Mějme množinu $M \subseteq \mathcal{V}$

    Budeme uvažovat průnik všech vektorových podprostorů vektorového prostoru $\mathcal{V}$,
    které obsahují $M$. Množinu která z těchto průniků vznikne označíme jako lineární obal
    množiny $M$.

    Lineární obal množiny $M$ budeme označovat jako $\langle M \rangle$
\end{definition}

\begin{definition}[Lineární kombinace]
    Mějme nějaké vektory:
    $$u_1, \ldots, u_n$$
    Potom můžeme uvažovat jiný vektor $v$ ve tvaru:
    $$v = c_1 \cdot u_1 + \ldots + c_n \cdot u_n$$
    Takovému vektoru $v$ potom říkáme lineární kombinace věktorů $u_1, \ldots, u_n$

    Zároveň platí, že vektor $v$ je lineárně zásivlý na věktorech $u_1, \ldots, u_n$
\end{definition}

\begin{theorem}[Rovnost množiny všech lin. kombinací a lineárního obalu]
    Mějme množinu $M \subseteq \mathcal{V}$

    Pro zjednodušení označme množinu všech lineárních kombinací
    vektorů z množiny $M$ jako $lcM$\footnote{Pouze dočasně, toto
    označení nebudeme běžně používat.}

    Potom tvrdíme:
    $$\langle M \rangle = lcM$$
\end{theorem}
\begin{proof}
    Chceme ukázat, že platí:
    $$\langle M \rangle \subseteq lcM \wedge \langle M \rangle \supseteq lcM$$

    Pomocné tvrzení: $lcM$ je vektorový podprostor:
    \begin{enumerate}
        \item Sečtením dvou lineárních kombinací z $lcM$ dostaneme opět lineární kombinaci z $lcM$.
        \item Vynásobením lineární kombinace z $lcM$ skalárem dostáváme lineární kombinaci z $lcM$.
    \end{enumerate}
    $lcM$ je tedy vektorový podprostor. A pro každý vektor $\vec{v} \in M$ určitě existuje
    lineární kombinace $\vec{c}$ taková, že $\vec{v} = \vec{c}$. $lcM$ tedy určitě
    obsahuje $M$

    Důkaz pro $\langle M \rangle \subseteq lcM$: plyne z toho, že $lcM$ je vektorový
    podprostor obsahujicí $M$ a z toho, že $<M>$ je průnik všech takových vektorových
    podprostorů. A průnik je určitě podmnožinou.

    Důkaz pro $\langle M \rangle \supseteq lcM$:
    $$\vec{v} \in lcM \Rightarrow \vec{v} = c_1
        \cdot \vec{u_1} + \ldots + c_n \cdot \vec{u_n}, \vec{u} \in M$$
    Potom:
    $$\vec{u_i} \in \mathcal{W}\; \forall \text{vektorové podprostory}
        \mathcal{W} \subseteq \mathcal{V}$$
    $$c_1 \cdot \vec{u_1} + \ldots + c_n \cdot \vec{u_n} \; \forall
    \text{vektorové podprostory} \mathcal{W} \subseteq \mathcal{V}$$
    to znamená, že $\vec{v} \in \bigcap\limits_{i} \mathcal{W}_i$
\end{proof}

Označení $lcM$ nebudeme nadále používat, protože jak jsme si ukázali, jedná se vlastně
o ekvivalentní definici lienárního obalu.

\begin{definition}[Lineárné nezávislost vektorů]
    Uvažujme množinu vektorů:
    $$(\vec{u_1}, \ldots, \vec{u_n})$$
    Řekneme, že vektory $\vec{u_1}, \ldots, \vec{u_n}$ jsou lineárně nezávislé,
    jestliže:
    $$c_1\cdot\vec{u_1}+\ldots +c_n\cdot\vec{u_n} = \vec{o} \Rightarrow \forall c_i = 0$$
\end{definition}

\begin{definition}[Báze vektorového podprostoru]
    Báze vektorového podprostoru $\mathcal{W}$ je uspořádaná n-tice $B$ lineárně
    nezávislých vektorů, které generují $\mathcal{W}$.
    Kde \textit{generují} znamená, že $\langle B \rangle = \mathcal{W}$.
\end{definition}

\begin{example}[Příklad báze vektorového podprostoru]
    $$\mathcal{W} = \mathcal{V} = \mathbb{R}^3$$
    $$\mathcal{B} = \big((1, 0, 0), (0, 1, 0), (0, 0, 1)\big)$$
    Je $\mathcal{B}$ báze vektorového podprostoru $\mathcal{W}$?
    \begin{enumerate}
        \item $(a, b, c) = a \cdot(1, 0, 0) + b \cdot (0, 1, 0) + c \cdot (0, 0, 1)$ \hfill
            $\mathcal{B}$ generuje celé $\mathcal{W}$
        \item $c_1 \cdot(1, 0, 0) + c_2 \cdot (0, 1, 0) + c_3 \cdot (0, 0, 1) = (0, 0, 0)
            \Rightarrow c_1 = 0, c_2 =, c_3 = 0$ \hfill $\mathcal{B}$ je lineárně nezávislé
    \end{enumerate}
\end{example}

\begin{definition}[Dimenze vektorového prostoru]
    Počet prvků báze $\mathcal{B}$ budeme označovat jako dimenzi vektorového
        prostoru $\langle B \rangle$
\end{definition}

\begin{example}[Báze vektorového prostoru matic] Uvažujme:
    $$\mathcal{V} = Mat_{2,3}(\mathbb{R})$$
    Potom bázi můžeme určit jako:
    \[
        \mathcal{B} = \Bigg(
            \begin{pmatrix}
                1 & 0 & 0 \\
                0 & 0 & 0
            \end{pmatrix},
            \begin{pmatrix}
                0 & 1 & 0 \\
                0 & 0 & 0
            \end{pmatrix},
            \begin{pmatrix}
                0 & 0 & 1 \\
                0 & 0 & 0
            \end{pmatrix},
            \begin{pmatrix}
                0 & 0 & 0 \\
                1 & 0 & 0
            \end{pmatrix},
            \begin{pmatrix}
                0 & 0 & 0 \\
                0 & 1 & 0
            \end{pmatrix},
            \begin{pmatrix}
                0 & 0 & 0 \\
                0 & 0 & 1
            \end{pmatrix}
            \Bigg)
    \]
    A vidíme, že báze tohoto vektorového prostoru je 6.

    Což dává smysl mimo jiné díky tomu, že v případě $Mat_{2,3}$ v podstatě pracujeme s
    $\mathbb{R}^6$ pouze s tím, že jsme prvky uspořádali do obdélníku. Tato změna uspořádání
    nemá z hlediska aditivní grupy a vektorového prostoru jako takového žádný zvláštní
    význam a změna se projeví až ve chvíli, kdy začneme matice násobit.
\end{example}

% 0:56 příklad na bázi C<0,1>

\subsubsection{Speciální zobrazení mezi vektorovými prostory}

\begin{definition}[Homomorfismus vektorovych prostorů]
    Uvažujme dva vektorové prostory $\mathcal{V}, \mathcal{W}$ a zobrazení $\varphi$:
    $$\varphi = \mathcal{V} \rightarrow \mathcal{W}$$
    kde $\varphi$ bude mít následujicí vlastnosti:
    \begin{enumerate}
        \item $\varphi(\vec{v_1} + \vec{v_2}) = \varphi(\vec{v_1}) + \varphi(\vec{v_2})\;
            \forall \vec{v_1}, \vec{v_2} \in \mathcal{V}$ \hfill Zachování součtu
        \item $\varphi(k\cdot v) = k \cdot \varphi (\vec{v}) \;
            \forall \vec{v} \in \mathcal{V} \; \forall k \in F$ \hfill Zachování násobení skalárem
    \end{enumerate}
    Neformálně řeceno $\varphi$ je zobrazení, které zachovává operace.

    Potom zobrazení $\varphi$ nazýváme homomorfismus\footnote{Někdy se také používá pojem
    lineární zobrazení} (vektorových prostorů).
    \label{def:homo}
\end{definition}

\begin{example}[Příklad homomorfismu vektorových prostorů]
    $$\mathcal{V} = \mathbb{R}^2, \mathcal{W} = \mathbb{R}^4$$
    $$\varphi(\vec{v}) = \vec{w}$$
    $$\vec{w} = (v_1, 0, v_1, v_1 + v_2)$$
    $$\varphi\big((1,2)\big) = (1, 0, 1, 3)$$

    Je takto definované $\varphi$ homomorfismus? Musí platit podmínky z definice
    homomorfismu \ref{def:homo}.

    První podmínka:
    $$LS = \varphi(\vec{u} + \vec{v}) = \varphi(u_1 + v_1, 0, u_1 + v_1 + u_2 + v_2),
    \vec{u}, \vec{v} \in \mathbb{R}^2$$
    $$PS = \varphi(\vec{u}) + \varphi(\vec{v}) = (u_1, 0, u_1 + u_2) + (v_1, 0, v_1 + v_2) =
    (u_1 + v_1, 0, u_1 + u_2 + v_1 + v_2), \vec{u}, \vec{v} \in \mathbb{R}^2$$
    $$LS = PS$$

    První podmínka tedy platí a stejným postupem by bylo možné ukázat i platnost
    druhé podmínky, jedná se tedy o homomorfismus.

    % \begin{enumerate}
    %     \item $LS = \varphi(\vec{u} + \vec{v}) = \varphi(u_1 + v_1, u_2 + v_2)$
    %         kde $\vec{u}, \vec{v} \in \mathbb{R}^2$
    %         $PS = \varphi(\vec{u}) = + $
    % \end{enumerate}
\end{example}

\begin{definition}[Jádro a obraz homomorfismu] Uvažujme vektorový prostor $\mathcal{V}$
    a homomorfismus $\varphi$, potom je kernel $\ker$ homomorfismu $\varphi$ definován takto:
    $$ker\,\varphi = \{\vec{v} \in \mathcal{V}; \varphi(\vec{v}) = \vec{o}\}$$
    A obraz $im$ homomorfismu $\varphi$ je definován takto:
    $$im\,\varphi = \{\vec{w} \in \mathcal{W}; \exists \vec{v} \in \mathcal{V}\,
        \text{tak, že}\, \varphi(\vec{v}) = \vec{w}\}$$
\end{definition}

\begin{theorem}[$ker\, \varphi$ a $im\, \varphi$ jsou vektorové podprostory $V$ a $W$
    v tomto pořadí]

    \begin{enumerate}
        \item[]
        \item $ker\,\varphi$ je vektorový podprostor $\mathcal{V}$,
            kde $\mathcal{V}$ je z definice \ref{def:homo}.
        \item $im\,\varphi$ je vektorový podprostor $\mathcal{W}$,
            kde $\mathcal{W}$ je z definice \ref{def:homo}.
    \end{enumerate}
\end{theorem}
\begin{proof}
    \begin{enumerate}
        \item[]
        \item $\vec{u}, \vec{v} \in ker\,\varphi:
            \varphi(\vec{u}) = \vec{o},\;\varphi(\vec{v}) = \vec{o}$ \\
            $\varphi(\vec{u} + \vec{v}) = \varphi(\vec{u}) + \varphi
            (\vec{u})\footnote{Vychází z vlastností homomorfismu.} =
            \vec{o} + \vec{o} = \vec{o}$
        \item $\vec{u} \in ker\,\varphi$ \\
              $\varphi(k \cdot \vec{u}) = k \cdot \varphi(u) \footnote{Z definice homomorfismu}
              = k \cdot \vec{o} = \vec{o}$
    \end{enumerate}
    Ukázali jsme, že $ker\,\varphi$ splňuje všechny podmínky k tomu, aby byl vektorový
    podprostor $\mathcal{V}$.

    Podobným způsobem bychom ukázali i druhou část věty o $im\,\varphi$ a došli také ke kladnému
    závěru.
\end{proof}

\begin{definition}[Monomorfismus]
    Jestliže je homomorfismus injektivní, nazýváme ho monomorfismus.
\end{definition}

\begin{definition}[Epimorfismus]
    Jestliže je homomorfismus surjektivní, nazýváme ho epimorfismus.
\end{definition}

\begin{definition}[Izomorfismus]
    Jestliže je homomorfismus bijektivní, nazýváme ho izomorfismus.
\end{definition}

\begin{definition}[Endomorfismus]
    Jestliže má homomorfismus $\varphi$ výchozí i cílovou množinu totožnou, tedy:
    $$\varphi: V \rightarrow V$$
    nazveme ho endomorfismus.

    Intuitivně můžeme říct, že se jedná se o homomorfismus
    do sebe sama.
\end{definition}

\begin{definition}[Automorfismus]
    Homomorfismu, který je endomorfismem a současně izomofismem nazveme automorfismus.
\end{definition}

\begin{theorem}
    $$\varphi: \mathcal{V} \rightarrow \mathcal{W}\,\text{je epimorfismus}\,\Leftrightarrow
    im\,\varphi = \mathcal{W}
    $$
\end{theorem}

\begin{theorem}
    $$\varphi: \mathcal{V} \rightarrow \mathcal{W}\,\text{je monomorfismem}\,\Leftrightarrow
    ker\,\varphi = \{\vec{o}\}
    $$
\end{theorem}
\begin{proof}
    Dokažme, že $\varphi: \mathcal{V} \rightarrow \mathcal{W}$ není monomorfismem
    $\Leftrightarrow ker\,\varphi \neq \{\vec{o}\}$

    Důkaz pro $\varphi: \mathcal{V} \rightarrow \mathcal{W}$ není monomorfismem
    $\Rightarrow ker\,\varphi \neq \{\vec{o}\}$:

    Předpokládejme, že $\varphi$ nená monomorfismus, to znamená, že $\varphi$ není inejktivní.

    To, že $\varphi$ není injektivní znamená, že existují nějaké vektory $\vec{u}, \vec{v} \in
    \mathcal{V}$, pro které:
    $$\vec{u \neq \vec{v}}, \varphi(\vec{u}) = \varphi(\vec{v})$$
    potom
    $$\varphi(\vec{u} - \vec{v}) =\footnote{Tato rovnost vychází z definice homomorfismu.} \varphi(\vec{u}) - \varphi(\vec{u}) = \vec{o}$$
    Ovšem $\vec{u}$ je různé od $\vec{v}$ a tedy:
    $$\vec{u} - \vec{v} \in ker\,\varphi$$
    A $\vec{u} - \vec{v}$ je nulový vektor, takže jádro je netriviální.

    Důkaz pro $\varphi: \mathcal{V} \rightarrow \mathcal{W}$ není monomorfismem
    $\Leftarrow ker\,\varphi \neq \{\vec{o}\}$:

    Předpokládejme, že $im\, \varphi$ je netriviální a ukážeme, že pak zobrazení nemůže být
    homomorfismem.

    Z předpokladu netriviálního jádra:
    $$\exists \vec{v} \neq \vec{o}, \vec{v} \in \mathcal{V}\,
        \text{tak, že}\,\varphi(\vec{v}) = \vec{o}$$
    $$\vec{u} \in \mathcal{V}\; \varphi(\vec{u}) = \vec{w} \in \mathcal{W}$$
    $$\varphi(\vec{u} + \vec{v}) = \varphi(\vec{u}) + \varphi(\vec{v}) = \vec{w} + \vec{o} = \vec{w}$$
\end{proof}

\subsection{Matice}

\begin{definition}[Stopa]
    Stopa je definována pro čtvercové matice. Jedná se o nějaké zobrazení,
    které čtvercové matici přiřadí jedno číslo. Stopu budeme značit
    $tr$\footnote{Z anglického trace.}

    $$tr: Mat_n(F) \rightarrow F$$
    $$tr(A) = \sum_{i=1}^na_{ii}$$

    Vlastnosti:
    \begin{itemize}
        \item $tr(A^T) = tr(A)$
        \item $tr(A + B) = tr(A) + tr(B)$
        \item $tr(k \cdot A) = k \cdot tr(A)$
        \item $tr(A\cdot B) = tr(B \cdot A)$\footnote{Zajímavé však je, že
        $tr(ABC) \neq tr(ACB)$}
    \end{itemize}

\end{definition}
\begin{proof}
    ($tr(A\cdot B) = tr(B \cdot A)$)

    $$C = AB, D = BA$$
    $$tr(AB) = \sum_{i=1}^n\sum_{k=1}^n a_{ik} \cdot b_{ki}$$
    $$tr(BA) = \sum_{i=1}^n\sum_{k=1}^n b_{ik} \cdot a_{ki} = \sum_{i=1}^n\sum_{k=1}^n a_{ki}
    \cdot b_{ik} = \sum_{k=1}^n\sum_{i=1}^n a_{ik} \cdot b_{ki} =
    \sum_{i=1}^n\sum_{k=1}^n a_{ik} \cdot b_{ki}$$

\end{proof}

\begin{definition}[Determinant]
    Determinant budeme definovat pro čtvercové matice.
    Opět se jedná o nějaké zobrazení, které čtvercové matici přiřadí
    jedno číslo. Determinant matice $A$ budeme značit $det\,A$, nebo také $|A|$.

    $$det: Mat_n(F) \rightarrow F$$
    $$det\, A = \sum_{\sigma \in S_n} sgn(\sigma) \cdot a_{1\sigma(1)} \cdot a_{2\sigma(2)}
    \cdot \ldots \cdot a_{n\sigma(n)}$$
\end{definition}

\begin{example}[Výpočet determinantu pro matice řádu 1]
    $$n = 1: S_1=\{1\}$$
    $$det\,A = a_{11}$$
\end{example}

\begin{example}[Výpočet determinantu pro matice řádu 2]
    $$n = 2: S_2=\{(1,2), (2,1)\}$$
    $$det\,A = a_{11} \cdot a_{22} - a_{12} \cdot a_{21}$$
\end{example}

\begin{example}[Výpočet determinantu pro matice řádu 3]
    $$n = 3: S_3=\{(1,2,3), (1,3,2), (2,1,3), (2,3,1), (3,1,2), (3,2,1)\}$$
    $$det\,A = a_{11} \cdot a_{22} - a_{12} \cdot a_{21}$$

    $$det\,A = a_{11}\cdot a_{22}\cdot a_{33} +
               a_{12}\cdot a_{23}\cdot a_{31} +
               a_{13}\cdot a_{21}\cdot a_{32} -
               a_{11}\cdot a_{23}\cdot a_{32} -
               a_{12}\cdot a_{21}\cdot a_{33} -
               a_{13}\cdot a_{22}\cdot a_{31}
    $$
\end{example}

\begin{definition}[Schodovitá matice]

\end{definition}


\section{Pátá přednáška}
\subsection{Metody pro výpočet determinantu}
Vzorec pro obecný výpočet determinantu \ref{def:determinant} vyžaduje
všechny permutace řádkových indexů, pro matici řádu $n$ je takových
permutací $n!$ a jejich počet tak roste velmi rychle s řádem matice.
Je proto komplikované tímto způsobem spočítat determinanty větších
matic.

Dost často se v praxi vyskytují matice, které jsou nějakým způsobem
speciální a to hlavně tím, že buďto obsahují řádek, případně
sloupec s hodně nulami, nebo nějakou jednoduchou úpravou můžeme
takový řádek s hodně nulami do matice dostat. V takovém případě
je možné použít následujicí metody a výpočet determinantu zefektivnit.

Pokud však takový řádek/sloupec s hodně nulami v matici neexistuje,
nebo je jich tam velice málo, aplikováním těchto metod si příliš nepomůžeme
a stále se bude jednat o problém s časovou náročností $n!$.
\subsubsection{Laplaceova metoda}
Laplaceova metoda rozvoje podle vybraného řádku, nebo podle
vybraného sloupce. Vybereme řádek/sloupec podle kterého chceme
rozvoj udělat, index tohoto řádku/sloupce označmě jako $i$ a
následně pomocí tohoto řádku výpočet determninantu matice řádu $n$ rozložit na součet $n$
determninantů matice řádu $n - 1$ vynásobené hodnotami z
řádku/sloupce podle kterého jsme rozvoj vytvářeli.

\[
|A| = (-1)^{i+1} \cdot a_{i1} \cdot M_{i1} + (-1)^{i+2}
\cdot a_{i2} \cdot M_{i2} + \ldots + (-1)^{i+n} \cdot a_{in} \cdot M_{in}
\]
\[
|A| = \sum_{j=1}^n (-1)^{i + j} \cdot a_{ij} \cdot M_{ij}
\]

Kde $M_{ij}$ představuje minor vzniklý vynecháním i-tého řádku a j-tého sloupce.

Mějme matici řádu 3:
\[
    A = \begin{pmatrix}
        \tikzmark{l1} a_{11} & a_{12} & a_{13}  \tikzmark{r1} \\
        a_{21} & a_{22} & a_{23} \\
        a_{31} & a_{32} & a_{33} \\
    \end{pmatrix}
    \DrawBox[dashed, red, rounded corners=0.5ex ]{l1}{r1}{\textcolor{red}{}}
\]

Vyberme pro aplikaci Laplaceova rozvoje první řádek a zamysleme
se, v jakých všech součinech bude figurovat jeho první člen $a_{11}$:
$$M_{a_{11}} = a_{11} \cdot a_{22} \cdot a_{33} - a_{11} \cdot a_{23} \cdot a_{32}$$
Když z $M_{a_{11}}$ vytkneme $a_{11}$, dostáváme:
\[
    M_{a_{11}} =
    a_{11} \cdot (a_{22} \cdot a_{33} - a_{23} \cdot a_{32}) =
    a_{11} \cdot
    \begin{vmatrix}
        a_{22} & a_{23} \\
        a_{32} & a_{33} \\
    \end{vmatrix}
\]
Totéž pro $a_{12}$:
    $$M_{a_{12}} = a_{12} \cdot a_{23} \cdot a_{31} - a_{12} \cdot a_{21} \cdot a_{33}$$
\[
    M_{a_{12}} =
    a_{12} \cdot (a_{23} \cdot a_{31} - a_{21} \cdot a_{33}) =
    - a_{12} \cdot
    \begin{vmatrix}
        a_{21} & a_{23} \\
        a_{31} & a_{33} \\
    \end{vmatrix}
\]

Proto:
\[
    det(A) =
        a_{11} \cdot
            \begin{vmatrix}
                a_{22} & a_{23} \\
                a_{32} & a_{33} \\
            \end{vmatrix} -
        a_{12} \cdot
        \begin{vmatrix}
            a_{21} & a_{23} \\
            a_{31} & a_{33} \\
        \end{vmatrix} +
        a_{13} \cdot
        \begin{vmatrix}
            a_{21} & a_{22} \\
            a_{31} & a_{32} \\
        \end{vmatrix}
\]

Znaménka u jednotlivých prvků plynou ze známének permutací ve výpočtu determinantu,
můžeme však použít pomůcku, která říka, že pokud součet indexů prvku pro který počítáme
minor bude sudý, bude mít znaménko $+$ a pokud bude součet lichý, bude mít znaménko $-$.

Postup, který byl naznačen na matici řádu $3$ funguje obecně pro libovolné matice řádu $n$.

\begin{example}[Výpočet determinantu pomocí Laplaceova rozvoje]
    \[
      d = \begin{vmatrix}
          2 & 1 & 5 & -1 & \tikzmark{l1}0 \\
          3 & 2 & 4 & 0 & 0  \\
          1 & 0 & -2 & 1 & 1 \\
          3 & 2 & 0 & 5 & 0  \\
          2 & 0 & -2 & 4 &  9 \tikzmark{r1}
      \end{vmatrix}
      \DrawBox[dashed, red, rounded corners=0.5ex ]{l1}{r1}{\textcolor{red}{}}
      % \DrawBox[thick, red, rounded corners=2ex ]{left1}{right1}{\textcolor{red}{\footnotesize$s^1$}}
    % \DrawBox[thick, blue, dashed]{left2}{right2}{\textcolor{blue}{\footnotesize$s^2$}}
    \]

    \[ d = 1 \cdot
    \begin{vmatrix}
        2 & 1 & 5 & -1 \\
        \tikzmark{l1} 3  & 2 & 4 & 0 \tikzmark{r1}\\
        3 & 2 & 0 & 5 \\
        2 & 0 & -2 & 4
        \DrawBox[dashed, red, rounded corners=0.5ex ]{l1}{r1}{\textcolor{red}{}}
    \end{vmatrix}
     + 9 \cdot
    \begin{vmatrix}
        2 & 1 & 5 & -1 \\
        \tikzmark{l1} 3  & 2 & 4 & 0 \tikzmark{r1}\\
        1 & 0 & -2 & 1 \\
        3 & 2 & 0 & 5
        \DrawBox[dashed, red, rounded corners=0.5ex ]{l1}{r1}{\textcolor{red}{}}
    \end{vmatrix}
    \]

    \[d =
        -3 \cdot
        \begin{vmatrix}
            1 & 5 & -1 \\
            2 & 0 & 5 \\
            0 & -2 & 4
        \end{vmatrix}
        + 2 \cdot
        \begin{vmatrix}
            2 & 5 & -1 \\
            3 & 0 & 5 \\
            2 & -2 & 4
        \end{vmatrix}
        - 4 \cdot
        \begin{vmatrix}
         2 & 1 & -1 \\
         3 & 2 & 5 \\
         2 & 0 & 4
        \end{vmatrix} + 9 \cdot \Bigg (
        -3 \cdot
        \begin{vmatrix}
            1 & 5 & -1 \\
            0 & -2 & 1 \\
            2 & 0 & 5
        \end{vmatrix}
        + 2 \cdot
        \begin{vmatrix}
            2 & 5 & -1 \\
            1 & -2 & 1 \\
            3 & 0 & 5 \\
        \end{vmatrix}
        - 4 \cdot
        \begin{vmatrix}
            2 & 1 & -1 \\
            1 & 0 & 1 \\
            3 & 2 & 5
        \end{vmatrix} \Bigg)
    \]
\end{example}

S použitím této metody je možné také vybrat více řádků/sloupců současně, což v některých
případech může být výhodné.

Mějme matici $A$ řádu $n$ a indexy $$i, j = 1, \ldots n$$
Potom vybereme řádky $$i_1 \ldots i_q, 1 \leq q \leq n$$
Potom:
$$det(A) =(-1)^{i_1 + \ldots i_q + 1 + \ldots + q}
A_{i_1, \ldots, i_q; 1,\ldots, q} M_{i_1, \ldots, i_q; 1,\ldots, q} + \ldots +
(-1)^{i_1, \ldots, i_q; n - q + 1,\ldots, n} \cdot
A_{i_1, \ldots, i_q; n - q + 1,\ldots, n} M_{i_1, \ldots, i_q; n - q + 1,\ldots, n}
$$

\begin{example}[Laplaceova metoda s výběrem více řádků/sloupců]
\[d =
    \begin{vmatrix}
        2 & 1 & 0 & 1 & 5 \\
        \tikzmark{l1}0 & 0 & 2 & 0 & 7 \tikzmark{r1}\\
        3 & 1 & 5 & 2 & -2 \\
        \tikzmark{l2} 1 & 0 & 0 & 0 & 0 \tikzmark{r2}\\
        2 & 5 & 2 & 3 & -8
    \end{vmatrix}
    \DrawBox[dashed, red, rounded corners=0.5ex ]{l1}{r1}{\textcolor{red}{}}
    \DrawBox[dashed, red, rounded corners=0.5ex ]{l2}{r2}{\textcolor{red}{}}
\]

\[d =
(-1)^{2 + 4 + 1 + 2} \cdot
\begin{vmatrix}
    0 & 0\\
    1 & 0
\end{vmatrix}
\cdot
\begin{vmatrix}
    0 & 1 & 5\\
    5 & 2 & -2\\
    2 & 3 & -8
\end{vmatrix}
\ldots
\]
Takových členů bude obecně $\binom{n}{q}$, kde $q$ označuje počet vybraných řádků/sloupců
v tomto případě tedy $\binom{5}{2}$. Díky vhodně vybraným řádkům pro vytvoření rozvoje však velká
část těchto členů bude ve výsledku nulová a ty můžeme tedy stejně jako v předchozím příkladu rovnou
vynechat a psát pouze ty nenulové a stejně tak můžeme rovnou vyjádřit hodnotu subdeterminantu,
pokud je zřejmá.
\[d =
-2 \cdot
\begin{vmatrix}
    1 & 1 & 5 \\
    1 & 2 & -2 \\
    5 & 3 & -8
\end{vmatrix}
-7 \cdot
\begin{vmatrix}
    1 & 0 & 1 \\
    1 & 5 & 2 \\
    5 & 2 & 3 \\
\end{vmatrix}
\]
\end{example}

\subsection{Inverzní matice}
\begin{definition}[Algebraický doplňek]
    Algebraický doplněk prvku $a_{ij}$ je $(-1)^{i+j}\cdot M_{ij}$.
\end{definition}

\begin{definition}[Adjungovaná matice]
    Adjungovaná matice $A^*$ vznikne z matice $A$ tak, že každý prvek
    nahradíme jeho algebraickým doplňkem.
\end{definition}

\begin{definition}[Inverzní matice]
    Inverzní matice, je matice k dané matici A, která splňuje:

    $$A \cdot A^{-1} = E = A^{-1} \cdot A$$

    Inverzní matici můžeme spočítat jako:
    $$A^{-1} = \frac{1}{|A|}\cdot (A^*)^{T}$$

    Z Cauchyho věty o součinu vyplývá, že determinant A musí být nenulový\footnote{Matice s nenulovým determinantem
    nazýváme regulární, a naopak matice s nulovým determinantem nazýváme singulární.}. % TODO add definition ref
\end{definition}

\begin{example}[Výpočet inverzní matice]
    Spočítejte inverzní matici $A^{-1}$ k matici $A$:
    \[A =
        \begin{pmatrix}
            1 & 2 & 0\\
            2 & -1 & 3 \\
            3 & 0 & 4 \\
        \end{pmatrix}
    \]
    Prvním krokem je zjistit, zda je $A$ vůbec regulární matice, tedy spočítat determinant $A$.
    $$|A| = 1 \cdot (-1) \cdot 4 + 3 \cdot 2 \cdot 3 - (4 \cdot 2 \cdot 2) = -2$$
    Následně potřebujeme vyjádřit adjungovanou matici $A^+$
    \[A^*=
        \begin{pmatrix}
            \begin{vmatrix}
                -1 & 3 \\
                0 & 4
            \end{vmatrix} &
            - \begin{vmatrix}
                2 & 3 \\
                3 & 4
            \end{vmatrix} &
            \begin{vmatrix}
                2 & -1 \\
                3 & 0
            \end{vmatrix} \\

            - \begin{vmatrix}
                2 & 0 \\
                0 & 4
            \end{vmatrix} &
            \begin{vmatrix}
                1 & 0 \\
                3 & 4
            \end{vmatrix} &
            - \begin{vmatrix}
                1 & 2 \\
                3 & 0
            \end{vmatrix} \\

            \begin{vmatrix}
                2 & 0 \\
                -1 & 3
            \end{vmatrix} &
            - \begin{vmatrix}
                1 & 0 \\
                2 & 3
            \end{vmatrix} &
            \begin{vmatrix}
                1 & 2 \\
                2 & -1
            \end{vmatrix}
        \end{pmatrix}
        = \begin{pmatrix}
            -4 & 1 & 3 \\
            -8 & 4 & 6 \\
            6 & -3 & -5
        \end{pmatrix}
    \]
    \[
    A^{-1} = \frac{1}{|A|} \cdot {A^*}^T = \frac{1}{-2} \cdot
        \begin{pmatrix}
            -4 & -8 & 6\\
            1 & 4 & -3 \\
            3 & 6 & -5
        \end{pmatrix}
        = \begin{pmatrix}
            2  & 4 & -3 \\
            \frac{-1}{2}  & -2 & \frac{3}{2} \\
            \frac{-3}{2} & -3 & \frac{5}{2}
        \end{pmatrix}
    \]
    Pro ověření:
    \[
      A \cdot A^{-1} =
      \begin{pmatrix}
          1 & 0 & 0\\
          0 & 1 & 0\\
          0 & 0 & 1
      \end{pmatrix}
    \]
 \end{example}

 Pro každou regulární matici jsme schopni tímto způsobem najít matici inverzní.
 Vezmeme li množinu regulárních matic řádu n a operaci násobení matic, zjistíme,
 že toto násobení je asociativní, ke každé matici existuje matice inverzní a vzhledem
 k násobení máme i neutrální prvek, kterým je jednotková matice. Regulární matice nad
 polem $F$ řádu $n$ s operací násobení matic tvoří grupu! Tuto grupu značíme jako:
 $$GL(n, F)$$

\begin{example}[Počet prvků grupy GL nad konečným polem]
    Kolik prvků má $GL(n, \mathbb{F}_{2^k})$?
\end{example}

\begin{example}[Nad polem $F_7$ řešte maticovou rovnici]
    $$X\cdot A = B$$
    Pro neznámou matici $X$ a
    \[
    A = \begin{pmatrix}
        2 & 3 \\
        5 & 1
    \end{pmatrix}, \;
    B = \begin{pmatrix}
        3 & 0 \\
        5 & 6
    \end{pmatrix}
    \]

Je nutné pamatovat na to, že násobení matic není komutativní a správně vynásobit obě
strany korektně inverzní maticí.
    \begin{align*}
        X \cdot A &= B \\
        X \cdot A \cdot A^{-1} &= B \cdot A^{-1}\\
        X &= B \cdot A^{-1}
    \end{align*}
Vyjádříme inverzní matici $A^{-1}$:
\[A^{-1} =\frac{1}{|A|} \cdot {A^*}^T = \frac{1}{1} \cdot
\begin{pmatrix}
    1 & 2 \\
    4 & 2
\end{pmatrix}^T = \begin{pmatrix}
    1 & 4 \\
    2 & 2
\end{pmatrix}
\]
A dosadíme do dříve získaného vztahu:
    \begin{align*}
        X &=
        \begin{pmatrix}
            3 & 0 \\
            5 & 6
        \end{pmatrix} \cdot
        \begin{pmatrix}
            1 & 4 \\
            2 & 2
        \end{pmatrix} \\
        X &=
        \begin{pmatrix}
        3 & 5 \\
        3 & 4
        \end{pmatrix}
    \end{align*}
\end{example}

\subsection{Hodnost matice}
Z definice \ref{def:lin_nezavislost} víme co je lineární nezávislost vektorů. V každé
matici můžeme řádky (případně sloupce) uvažovat jako vektory. Může nás zajímat, kolik je v
matici lineárně nezávislých řádků (případně sloupců).

Počet lineárně nezávislých řádků se nazývá hodnost matice. Hodnost matice je tedy počet jejich
lineárně nezávislých řádků, což je totéž jako počet lineárně nezávislých sloupců. Hodnost
matice $A$ značíme:
$$h(A)$$

Nejmenší hodnost matice může být 0, hodnost 0 má pouze nulová matice, protože nulový vektor
není nezávislý ani sám o sobě. A maximální možná hodnost pro matici o rozměrech
$m \times n$ je $min(m, n)$.
$$0 \leq h(A) \leq min(m, n)$$


\subsubsection{Výpočet hodnosti matice}
Pro jednotlivé řádky by bylo možné využít vztahu z definice \ref{def:lin_nezavislost},
pomocí něj sestavit soustavu lineárních rovnic a vypočítat její řešení. To však může být
zbytečně zdlouhavé, budeme tedy používat tzv. ekvivalentní (elementární) úpravy matice,
které nemění hodnost matice.

Ekvivalentními úpravami matice jsou:
\begin{enumerate}
    \label{ekv_upravy}
    \item Přičtení k-násobku řádku k jinému řádku.
    \item Vynásobení řádku nějakým nenulovým číslem $l$.
    \item Výměna řádků\footnote{Výměna řádků jde relizovat pomocí prvních dvou úprav,
    její uvádění zde tedy není nutné.}.
\end{enumerate}

Pomocí těchto ekvivalentních úprav upravíme matici na schodovitou. A hodnost schodovité
matice je velice snadné spočítat, protože u schodovité matice je její hodnost počet nenulových
řádků (vektorů).

Maximální hodnost matice označujeme jako plnou hodnost.

\begin{example}[Výpočet hodnosti matice]
    Vypočtěte hodnost matice $Q$.
    \[Q =
    \begin{pmatrix}
        1 & 2 & 0 & 5 \\
        2 & -1 & 2 & -3\\
        2 & 1 & -3 & 6\\
        6 & -1 & 1 & 0
    \end{pmatrix}
    \]

    Pomocí elementárních úprav převedeme matici $Q$ do schodovítého tvaru.
    \begin{align*}
        Q \sim
        \begin{pmatrix}
            1 & 2 & 0 & 5 \\
            0 & -5 & 2 & -13\\
            0 & -3 & -3 & -4\\
            0 & -13 & 1 & -30
        \end{pmatrix} \sim
        \begin{pmatrix}
            1 & 2 & 0 & 5 \\
            0 & -5 & 2 & -13\\
            0 & 15 & 15 & 20\\
            0 & 65 & -5 & 150
        \end{pmatrix} \sim
        \begin{pmatrix}
            1 & 2 & 0 & 5 \\
            0 & -5 & 2 & -13\\
            0 & 0 & 21 & -19\\
            0 & 0 & 21 & -19
        \end{pmatrix} \sim
        \begin{pmatrix}
            1 & 2 & 0 & 5 \\
            0 & -5 & 2 & -13\\
            0 & 0 & 21 & -19\\
            0 & 0 & 0 & 0
        \end{pmatrix} = R
    \end{align*}

    Z upravené matice vidímě, že:
    $$h(Q) = h(R) = 3$$
\end{example}

\subsection{Výpočet determinantu pomocí ekvivalentních úprav}
S využitím ekvivalentních úprav \ref{ekv_upravy} můžeme matici převést na
schodovitý tvar a jednoduše spočítat determinant schodovité matice jako
součin prvků na hlavní diagonále.

Při aplikaci úprav však musíme dávat pozor na to, jakým způsobem která úprava mění
determinant upravené matice:

\begin{enumerate}
    \item Přičtení k-násobku řádku k jinému řádku. \hfill Determinant se nemění.
    \item Vynásobení řádku nějakým nenulovým číslem $l$. \hfill Determinant bude $l$ krát větší.
    \item Výměna řádků.\hfill Změní se znaménko determinantu.
\end{enumerate}

Vliv těchto úprav na determinant je lehké ukázat na matici řádu 2. Platí však obecně pro
jakoukoliv matici řádu $n$.

Z těchto pravidel lze také vyvodit, že každá čtvercová matice řádu $n$ s plnou hodností
je regulární a s jakoukoliv menší než plnou hodností je singulární.

\begin{example}[Výpočet determinantu pomocí ekvivalentních úprav]
    Pomocí ekvivalentních úprav spočtěte determinant matice M.
    \[M=
    \begin{pmatrix}
        1 & 2 & 3 & 4 \\
        2 & 4 & 6 & 2 \\
        0 & 0 & 5 & 6 \\
        0 & 1 & 10 & 1
    \end{pmatrix}
    \eqop{r_2 \cdot  \frac{1}{2}}
    \begin{pmatrix}
        1 & 2 & 3 & 4 \\
        1 & 2 & 3 & 1 \\
        0 & 0 & 5 & 6 \\
        0 & 1 & 10 & 1
    \end{pmatrix} \eqop{r_2 - r_1}
    \begin{pmatrix}
        1 & 2 & 3 & 4 \\
        0 & 0 & 0 & -3 \\
        0 & 0 & 5 & 6 \\
        0 & 1 & 10 & 1
    \end{pmatrix} \eqop{r_2 \leftrightarrow r_4}
    \begin{pmatrix}
        1 & 2 & 3 & 4 \\
        0 & 1 & 10 & 1 \\
        0 & 0 & 5 & 6 \\
        0 & 0 & 0 & -3
    \end{pmatrix} = N
    \]

    \begin{align*}
        det(N) &= -15\\
        det(N) &= \frac{1}{2} \cdot -1 \cdot det(M)\\
        det(M) &= -2 \cdot det(N) = 30
    \end{align*}
\end{example}

\begin{definition}[Symetrická matice]
    Matice $A$ je symetrická, pokud platí:
    $$A^T = A$$
    Pro jednotlivé prvky matice musí tedy platit:
    $$a_{ij} = a_{ji}$$
    Symetrickou matici značíme s dolním indexem $sym$, např. $A_{sym}$
\end{definition}

\begin{definition}[Antisymetrická matice]
    Matice $A$ je antisymetrická, pokud platí:
    $$A^T = -A$$
    Pro jednotlivé prvky matice musí tedy platit:
    $$a_{ij} = -a_{ji}$$
    Antisymetrickou matici značíme s dolním indexem $alt$, např. $A_{alt}$
\end{definition}

\begin{theorem}[Rozklad čtvercové matice na symetrickou a antisymetrickou matici]
    Každou čtvercovou matici lze rozložit na součet matice symetrické a antisymetrické.

    $$A = A_{sym} + A_{alt}$$
\end{theorem}
\begin{proof}
    Matice $A_{sym}$ a $A_{alt}$ můžeme vždy zkonstruovat následovně:
    $$A_{sym} = \frac{1}{2} \cdot (A + A^T)$$
    Takto zvolená matice $A_{sym}$ je určitě symetrická, protože:
    $$(A_{sym})^T = \frac{1}{2}(A^T + (A^T)^T) = A_{sym}$$

    Podobně pro $A_{alt}$:
    $$A_{alt} = \frac{1}{2}\cdot (A - A^T)$$
    Takto zvolená matice $A_{alt}$ je určitě antisymetrická, protože:
    $$(A_{alt})^T = \frac{1}{2} \cdot (A^T - (A^T)^T) = - A_{alt}$$

    Zároveň vidíme, že výsledkem součtu těchto dvou matic je opravdu původní matice:
    $$A_{sym} + A_{alt} = \frac{1}{2} \cdot (A + A^T) + \frac{1}{2}\cdot (A - A^T) = A$$
\end{proof}

\subsection{Soustavy linearních rovnic}
Soustavy lineárních rovnic můžeme řešit elementárně (vyjadřovat neznámé a dosazovat
do ostatních rovnic, případně sečíst dvě rovnice a tím nějakou neznámou eliminovat). Tento
způsob však má nevýhodu, že není algoritmický.

Proto budeme používat postupy pomocí matic, které jsou jednoduše algoritmizovatelné.

Soustava lineárních rovnic obecně:
\begin{align*}
    a_{11} \cdot x_1 + \ldots a_{1n} \cdot x_n &= b_1\\
    \vdots\\
    a_{m1} \cdot x_1 + \ldots a_{mn} \cdot x_n &= b_m\\
\end{align*}
Proměnným $a_{ij}$ říkáme koeficienty, které tvoří obdélníkovou matici
o $m$ řádcích a $n$ sloupcích. Proměnné $x_1, \ldots, x_n$ nazýváme neznámé, ty budeme
počítat. A proměnné $b_1, \ldots, b_m$ nazýváme absolutní členy.
Tuto obecnou soustavu lineárních rovnic můžeme zapsat maticově:
\[
    \begin{pmatrix}
        a_{11} & \ldots & a_{1n} \\
        \vdots & \vdots & \vdots\\
        a_{m1} & \ldots & a_{mn}
    \end{pmatrix}
    \begin{pmatrix}
        x_1 \\
        \vdots \\
        x_m
    \end{pmatrix}
    =
    \begin{pmatrix}
        b_1 \\
        \vdots \\
        b_m
    \end{pmatrix}
\]

Někdy také zapisujeme jako:
$$A \cdot \vec{x} = \vec{b}$$

\subsubsection{Gaussova eliminační metoda}
Matici soustavy\footnote{Tuto matici tvoří jednotlivé koeficienty.} $A$ a vektor absolutních členů
$\vec{b}$ zapíšeme do jedné rozšířené matice následovně:
\begin{align*}
    \begin{pmatrix}[c|c]
        A & \vec{b}
    \end{pmatrix} =
    \begin{pmatrix}[ccc|c]
        a_{11} & \ldots & a_{1n} & b_1\\
        \vdots & \vdots & \vdots & \vdots \\
        a_{m1} & \ldots & a_{mn} & b_m
    \end{pmatrix}
\end{align*}
Tuto rozšířenou matici poté pomocí elementárních úprav upravujeme do vhodného (schodovitého) tvaru,
ze kterého dokážeme lehce zjistit řešení celé soustavy. Ze schodovité matice pak jednoduše dostaneme
řešení celé soustavy "zpětným chodem" od posledního řádku, kdy postupně zjišťujeme hodnoty neznámých.

Soustava rovnice je řešitelná právě tehdy, když:
\[
    h(A) = h\Bigg (
    \begin{pmatrix}[c|c]
        A & \vec{b}
    \end{pmatrix} \Bigg )
\]
V opačném případě matice řešitelná není.

V případě, že se hodnosti rovnají, a soustava je tedy řešitelná, mohou nastat dva případy:
\begin{enumerate}
    \item $h(A) = n$, kde n je počet neznámých \hfill Rovnice má právě jedno řešení.
    \item $h(A) = k < n$ \hfill Rovnice má nekonečně mnoho řešení\footnote{A tato řešení jsou
    závislá na $n-k$ libovolných parametrech. V případě, že navíc pracujeme nad konečným polem, není jich
    tedy nekonečně mnoho}.
\end{enumerate}
Mohou tedy nastat pouze 3 případy:
\begin{enumerate}
    \item Soustava není řešitelná.
    \item Soustava je řešitelná a má právě jedno řešení.
    \item Soustava je řešitelná a má nekončně mnoho řešení.
\end{enumerate}

\begin{example}[Příklad neřešitelné soustavy]
    Řešte následujicí soustavu rovnic:
    \begin{align*}
        x + y + z &= 2 \\
        x - y + 2z &= 1 \\
        7x + 4y + 7z &= -3
    \end{align*}

    Tuto soustava přepíšeme do matice a upravíme do schodovitého tvaru:
    \[
        \begin{pmatrix}[ccc|c]
            1 & 1 & 1 & 2 \\
            2 & -1 & 2 & 1 \\
            7 & 4 & 7 & -3
        \end{pmatrix} \eqop{r_2 - 2\cdot r_1, r_3 - 7\cdot r_1}
        \begin{pmatrix}[ccc|c]
            1 & 1 & 1 & 2 \\
            0 & -3 & 0 & -3 \\
            0 & -3 & 0 & -17
        \end{pmatrix} \eqop{r_3 - r_2}
        \begin{pmatrix}[ccc|c]
            1 & 1 & 1 & 2 \\
            0 & -3 & 0 & -3 \\
            0 & 0 & 0 & -14
        \end{pmatrix}
    \]
    Hodnost základní matice je v tomto případě zjevně 2 a hodnost
    celé rozšířené matice je zjevně 3. Hodnosti se nerovnají a soustava tedy nemá řešení.
\end{example}

\newpage
\bibliographystyle{czechiso}
\bibliography{bibliography}
\def\refname{Reference}
Přednášky SLA 1 - 4, přednášejicí: Kureš Miroslav, Doc. RNDr., Ph.D.
\end{document}
