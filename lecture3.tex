\section{Třetí přednáška}

\subsection{Vektorový prostor}
\begin{definition}[Vektorový prostor]

    Vektorový prostor je především Abelovská grupa
    $$(\mathcal{V}, +)$$
    Kromě operace + budeme mít další operaci vztaženou k nějakému poli $(F, +, \cdot)$
    a to operaci
    $$\cdot: F \times \mathcal{V} \rightarrow \mathcal{V}$$
    této operaci budeme říkat násobení skalárem a bude mít následujicí vlastnosti:
    \begin{enumerate}
        \item $a \cdot (\vec{u} + \vec{v}) = a \cdot \vec{u} + a \cdot \vec{v}, \forall
            a \in F \; \forall \vec{u}, \vec{v} \in \mathcal{V}$
        \item $(a + b) \cdot \vec{u} = a \cdot \vec{u} + b \cdot \vec{u},
            \forall a,b \in F \; \forall \vec{u} \in V$
        \item $(a \cdot b ) \cdot \vec{u} = a \cdot (b \cdot \vec{u}),
            \forall a,b \in F \; \forall \vec{u} \in V$
        \item $1 \cdot \vec{u} = \vec{u}, \forall \vec{u} \in \mathcal{V}$
    \end{enumerate}
    Tedy pokud máme nějaké pole $F$, nějakou abelovskou grupu $\mathcal{V}$ a definovaný
    skalární součin $\cdot$ s uvedenými vlastnostmi. Potom řekneme, že $\mathcal{V}$ je
    vektorovým prostorem nad polem $F$.

    Prvkům $\mathcal{V}$ pak budeme říkat vektory. Prvkům $F$ často říkáme skaláry,
    nebo čísla.
    \label{def:vector_space}
\end{definition}
\begin{example}[Příklad vektorového prostoru]
    $$F = \mathbb{R}, \mathcal{V} = \mathbb{R}^3$$
    $$\vec{u} + \vec{v} = (u_1 + v_1, u_2 + v_2, u_3 + v_3)
    \forall \vec{u}, \vec{v} \in \mathcal{V}$$
    Je $(\mathcal{V}, +)$ abelovská grupa?
    \begin{enumerate}[start=0]
        \item Operace $+$ je zjevně uzavřená na množině $\mathbb{R}^3$.
        \item Asociativita vychází z asociativity sčítání v $\mathbb{R}$.
        \item Neutrálním prvkem je: $\vec{o} = (0, 0, 0)$.
        \item Inverzní prvek k prvku $u$ dostaneme jako: $-\vec{u} = (-u_1, -u_2, -u_3)$.
        \item Komutativita je v tomto případě také jasná.
    \end{enumerate}
    $(\mathcal{V}, +)$ je tedy abelovská grupa\footnote{Otázka k zamyšlení je, zda by bylo
    možné operaci $+$ zavést nějak jinak a stále dodržet všechny požadované vlastnosti.}.

    operaci násobení skalárem $\cdot$ zavedeme takto:
    $$a \cdot (u_1, u_2, u_3) = (a \cdot u_1, a \cdot u_2, a \cdot u_3)$$
    Je zřejmé, že tato operace splňuje všechny požadované vlastnosti z definice
    \ref{def:vector_space}. Pojďme ověřit například třetí axiom:
    \begin{proof} Platnost třetí vlasnosti:
        $$LS: (a \cdot b) \cdot (u_1, u_2, u_3) =
        (a \cdot b \cdot u_1, a \cdot b \cdot u_2, a \cdot b \cdot u_3)$$
        $$PS: a\cdot (b \cdot (u_1, u_2, u_3)) = a \cdot (b \cdot u_1, b \cdot u_2, b \cdot u_3) =
        (a \cdot b \cdot u_1, a \cdot b \cdot u_2, a \cdot b \cdot u_3)$$
        $$LS = PS$$
    \end{proof}
    \label{ex:vec_space}
\end{example}

Ověření platnosti požadovaných vlastností bylo v případě \ref{ex:vec_space} triviální.
Podobně jednoduchý postup by šel použít i pro případy, kdy obecně
$\mathcal{V} = \mathbb{R}^n, \forall n \in \mathbb{N}$. Existují však příklady, ke kterým
se dostaneme později, kdy ověření platnosti není tak jednoduché.

\begin{definition}[Triviální vektorový prostor]
    Uvažujme libovolné pole $F$ a abelovskou grupu
    $$(V, +), \mathcal{V} = \{\vec{o}\}$$
    Což je zároveň minimální případ grupy, protože ta z definice vždy musí
    obsahovat minimálně jeden prvek, kterým je neutrální prvek.

    A operaci násobení skalárem zavedeme takto:
    $$a \cdot \vec{o} = \vec{o}, a \in F, \vec{o} \in \mathcal{V}$$
    Opět budou splněny všechny požadované vlastnosti, jedná se tedy o vektorový prostor
    a tento vektorový prostor budeme označovat jako triviální vektorový prostor.
\end{definition}

Dalším jednoduchý příklad vektorového prostorů:
libovolné pole $F$, $\mathcal{V} = F^n, n \in \mathbb{N}$

\begin{example}[Je $\mathbb{R}$ vektorový prostor nad $\mathbb{Q}$?]
    $$F = \mathbb{Q}, \mathcal{V} = \mathbb{R}$$
    Víme, že $\mathbb{R}$ s běžnou operací $+$ tvoří Abelovskou grupu.
    Násobení skalárem zavedeme jako běžné násobení.
    To že platí vlastnosti z definice \ref{def:vector_space} plyne z vlastností
    bežné operace násobené $\cdot$ na množině $\mathbb{R}$.

    Takže ano, v tomto případě se jedná o vektorový prostor.
\end{example}

Další příklady vektorových prostorů:
\begin{itemize}
    \item $F = \mathbb{R}, \mathcal{V} = C^0\langle a, b \rangle$
        \footnote{$C^n\langle a,b \rangle$ značí množinu funkcí na reálném intervalu
        $\langle a,b \rangle$, které jsou spojité až v n-té derivaci, $n \in \mathbb{N}$.}
    \item $F = \mathbb{R}, \mathcal{V} = C^1\langle a, b \rangle$
    \item $F = \mathbb{R}, \mathcal{V}=$ množina matic o velikosti $n \times n$
    \item $F = \mathbb{Q}, \mathcal{V} = \mathbb{R}$
    \item $F = \mathbb{R}, \mathcal{V} = \mathbb{C}$
    \item A dokonce i $F = \mathbb{C}, \mathcal{V} = \mathbb{R}$, v tomto případě však bude
        vytvoření vhodných operací netriviální.
\end{itemize}

\begin{definition}[Vektorový podprostor]
    Nechť $\mathcal{V}$ je vektorový prostor
    $$\mathcal{W} \subseteq \mathcal{V}$$
    A zároveň platí tyto vlastnosti:
    \begin{enumerate}
        \item $\vec{u}, \vec{v} \in \mathcal{W} \Rightarrow \vec{u} + \vec{v} \in \mathcal{W}$
        \item $\vec{u} \in \mathcal{W}, a \in F \Rightarrow  a \cdot \vec{u} \in \mathcal{W}$
    \end{enumerate}

    Pokud je $\mathcal{W}$ podmnožina $\mathcal{V}$ a zároveň splňuje dvě výše uvedené vlastnosti,
    potom řekneme, že $\mathcal{W}$ je vektorovým podprostorem vektorového prostu $\mathcal{V}$.
\end{definition}

\begin{example}[Ověření vektorového podprostoru]
    Mějme
    $$F = \mathbb{R}, \mathcal{V} = \mathbb{R}^3$$
    $$\mathcal{W}_1 = \{(a, 2\cdot a, 1), a \in \mathbb{R}\}$$

    Je $\mathcal{W}_1$ vektorový podprostor $\mathcal{V}$?
    \begin{enumerate}[start=0]
        \item Je zřejmé, že $\mathcal{W}_1 \subseteq \mathbb{R}^3$
        \item $(a, 2 \cdot a, 1) + (b, 2 \cdot b, 1) = (a + b, 2 \cdot a + 2 \cdot b, 2)
        \notin \mathcal{W}_1$
    \end{enumerate}
    První podmínka tedy není splněna a $\mathcal{W}_1$ v tomto případě není vektorovým podprostorem
    vektorového prostoru $\mathcal{V}$.

    Nyní uvažme $\mathcal{W}_2$ definované následovně:
    $$\mathcal{W}_2 = \{(a, 2\cdot a, 0), a \in \mathbb{R}\}$$
    \begin{enumerate}[start=0]
        \item Opět je zřejmé, že $\mathcal{W}_2 \subseteq \mathbb{R}^3$
        \item $(a, 2 \cdot a, 0) + (b, 2 \cdot b, 0) = (a + b, 2 \cdot (a + b), 0) = (c, 2 \cdot c, 0) \in \mathcal{W}$
        \item $a \cdot (b, 2b, 0) = (a \cdot b, 2\cdot a \cdot b, 0) = (c, 2c, 0) \in \mathcal{W}$
    \end{enumerate}
    Všechny požadované vlastnosti platí a $\mathcal{W}$ je tedy vektorový podprostor vektorového
    prostoru $mathcal{V}$.
\end{example}

\subsection{Matice}

Intuitivně můžeme matici definovat jako čísla, která jsou uspořádaná do obdélníkového schématu.

\begin{definition}[Matice]
    Uvažujme zobrazení $a$ definované takto:
    $$a = \{1 \ldots m\} \times \{1 \ldots n\} \rightarrow F$$

    Toto zobrazení vlastně přiřazuje dvojici indexů (sloupcový a řádkový) hodnotu,
    která se v matici nachází na daném indexu. Indexy budeme značit dolním indexem
    jako $a_{mn}$
\end{definition}

Psát matice jako zobrazení by bylo silně nepraktické, proto nadále budeme využívat
právě ono uspořádání čísel do obdélníkového schématu a označovat je velkými písmeny.

Prvky $a_{11}, a_{22}, \ldots, a_{kk}, k = min(m, n)$ budeme označovat jako hlavní
diagonálu.

Dále budeme často používat indexování pomocí $i, j$, kde
$$i = 1, \ldots, m$$
$$j = 1, \ldots, n$$

\begin{example}[Matice]

    Mějme matici A definovanou takto:
    $$ A = \begin{pmatrix}
        3 & 2 & 1 & -5 \\
        2 & \sqrt{2} & 3 & -\frac{1}{3}
        \end{pmatrix}  $$

    Potom $a(2, 2) = a_{22} = \sqrt{2}$, $a(2, 4) = a_{24} = -\sqrt{1}{3}$
\end{example}

\subsection{Speciální případy matic}
Mezi maticemi rozlišujeme celou řadu speciálních případů. Některé z nich jsou pouze
pro případ matic čtvercových, ale některé se dají definovat i obecně pro obdélníkové matice.

\begin{definition}[Horní trojúhelníková matice]
    Jestliže jsou v matici $A$ všechny prvky pod hlavní diagonálou 0,
    řekneme že matice $A$ je horní trojúhelníková.

    Řečeno formálně, v matici musí platit:
    $$i > j \Rightarrow a_{ij} = 0$$

    Podobně je definovaná dolní trojúhelníková matice, pouze se $i < j$ změni na $i > j$.
\end{definition}

\begin{definition}[Diagonální matice]
    Jestliže jsou v matici $A$ všechny prvky mimo hlavní diagonálou 0,
    řekneme že matice $A$ je horní diagonální.

    Řečeno formálně, v matici musí platit:
    $$i \neq j \Rightarrow a_{ij} = 0$$
\end{definition}

\begin{definition}[Nulová matice]
    Jestliže jsou v matici $A$ všechny prvky 0,
    řekneme že matice $A$ je nulová.

    Řečeno formálně, v matici musí platit:
    $$\forall i,j: a_{ij} = 0$$
\end{definition}

\begin{definition}[Jednotková matice]
    Jestliže je matice $A$ diagonální a zároveň jsou všechny hodnoty na hlavní diagonále 1,
    řekneme, že matice $A$ je jednotková.

    Řečeno formálně, v matici musí platit:
    \[
        a_{ij} = \delta_{ij}\footnote{Kroneckerovo delta} =
        \left\{
            \begin{array}{ll}
                1 & \text{ pro } i = j\\
                0 & \text{ pro } i \neq j\\
            \end{array}
        \right.
    \]

    Jednotkovou matici budeme značit $E$.
\end{definition}

\subsection{Základní operace na maticích}

Množinu všech matic o $m$ řádcích a $n$ sloupcích nad polem F
budeme označovat následovně $$Mat_{m,n}(F)$$

\begin{definition}[Součet matic]
    $$+: Mat_{m,n}(F) \times Mat_{m,n}(F) \rightarrow Mat_{m,n}(F)$$
    $$c_{ij} = a_{ij} + b_{ij}\; \forall i,j$$

    Tato operace je uzavřená, asociativní, má definovaný neutrální i inverzní prvek a navíc
    je komutativní.

    A tvoří tedy Abelovskou grupu $(Mat_{m,n}(F), +)$
    \label{def:mat_plus}
\end{definition}

\begin{definition}[Násobení matice skalárem]
    $$\cdot: F \times Mat_{m,n}(F) \rightarrow Mat_{m,n}(F)$$
    $$b_{ij} = k \cdot a_{ij}$$

    Tato operace splňuje požadavky pro operaci násobení skalárem z
    definice vektorového prostoru \ref{ex:vec_space}.

    A $Mat_{m,n}(F)$ v kombinaci s operací $+$ z definice \ref{def:mat_plus} a s
    touto operací $\cdot$ tvoří vektorový prostor:
    $$(Mat_{m,n}(F), +, \cdot)$$
\end{definition}

\begin{definition}[Transpozice matice]
    $$Mat_{m,n}(F) \rightarrow Mat_{m,n}(F)$$
    $$B = A^T$$
    $$b_{ji} = a_{ij}$$
    $$(A^T)^T = A$$
    $$(A + B)^T = A^T + B^T$$
\end{definition}

\begin{definition}[Násobení matic]
    $$\cdot: Mat_{m,n}(F) \times Mat_{m,p}(F) \rightarrow Mat_{m,p}(F)$$
    $$c_{ij} = \sum_{k=1}^n a_{ik} \cdot b_{kj}$$

    Tato operace nemusí být vždy definována (musí správně sedět dimenze).
    Je to operace asociativní. Ale není komutativní.

    Omezíme li se na čtvercové regulární matice řádu $n$, dostaneme v kombinaci s touto operací
    strukturu grupy.
\end{definition}

