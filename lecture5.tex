\section{Pátá přednáška}
\subsection{Metody pro výpočet determinantu}
Vzorec pro obecný výpočet determinantu \ref{def:determinant} vyžaduje
všechny permutace řádkových indexů, pro matici řádu $n$ je takových
permutací $n!$ a jejich počet tak roste velmi rychle s řádem matice.
Je proto komplikované tímto způsobem spočítat determinanty větších
matic.

Dost často se v praxi vyskytují matice, které jsou nějakým způsobem
speciální a to hlavně tím, že buďto obsahují řádek, případně
sloupec s hodně nulami, nebo nějakou jednoduchou úpravou můžeme
takový řádek s hodně nulami do matice dostat. V takovém případě
je možné použít následujicí metody a výpočet determinantu zefektivnit.

Pokud však takový řádek/sloupec s hodně nulami v matici neexistuje,
nebo je jich tam velice málo, aplikováním těchto metod si příliš nepomůžeme
a stále se bude jednat o problém s časovou náročností $n!$.
\subsubsection{Laplaceova metoda}
Laplaceova metoda rozvoje podle vybraného řádku, nebo podle
vybraného sloupce. Vybereme řádek/sloupec podle kterého chceme
rozvoj udělat, index tohoto řádku/sloupce označmě jako $i$ a
následně pomocí tohoto řádku výpočet determninantu matice řádu $n$ rozložit na součet $n$
determninantů matice řádu $n - 1$ vynásobené hodnotami z
řádku/sloupce podle kterého jsme rozvoj vytvářeli.

\[
|A| = (-1)^{i+1} \cdot a_{i1} \cdot M_{i1} + (-1)^{i+2}
\cdot a_{i2} \cdot M_{i2} + \ldots + (-1)^{i+n} \cdot a_{in} \cdot M_{in}
\]
\[
|A| = \sum_{j=1}^n (-1)^{i + j} \cdot a_{ij} \cdot M_{ij}
\]

Kde $M_{ij}$ představuje minor vzniklý vynecháním i-tého řádku a j-tého sloupce.

Mějme matici řádu 3:
\[
    A = \begin{pmatrix}
        \tikzmark{l1} a_{11} & a_{12} & a_{13}  \tikzmark{r1} \\
        a_{21} & a_{22} & a_{23} \\
        a_{31} & a_{32} & a_{33} \\
    \end{pmatrix}
    \DrawBox[dashed, red, rounded corners=0.5ex ]{l1}{r1}{\textcolor{red}{}}
\]

Vyberme pro aplikaci Laplaceova rozvoje první řádek a zamysleme
se, v jakých všech součinech bude figurovat jeho první člen $a_{11}$:
$$M_{a_{11}} = a_{11} \cdot a_{22} \cdot a_{33} - a_{11} \cdot a_{23} \cdot a_{32}$$
Když z $M_{a_{11}}$ vytkneme $a_{11}$, dostáváme:
\[
    M_{a_{11}} =
    a_{11} \cdot (a_{22} \cdot a_{33} - a_{23} \cdot a_{32}) =
    a_{11} \cdot
    \begin{vmatrix}
        a_{22} & a_{23} \\
        a_{32} & a_{33} \\
    \end{vmatrix}
\]
Totéž pro $a_{12}$:
    $$M_{a_{12}} = a_{12} \cdot a_{23} \cdot a_{31} - a_{12} \cdot a_{21} \cdot a_{33}$$
\[
    M_{a_{12}} =
    a_{12} \cdot (a_{23} \cdot a_{31} - a_{21} \cdot a_{33}) =
    - a_{12} \cdot
    \begin{vmatrix}
        a_{21} & a_{23} \\
        a_{31} & a_{33} \\
    \end{vmatrix}
\]

Proto:
\[
    det(A) =
        a_{11} \cdot
            \begin{vmatrix}
                a_{22} & a_{23} \\
                a_{32} & a_{33} \\
            \end{vmatrix} -
        a_{12} \cdot
        \begin{vmatrix}
            a_{21} & a_{23} \\
            a_{31} & a_{33} \\
        \end{vmatrix} +
        a_{13} \cdot
        \begin{vmatrix}
            a_{21} & a_{22} \\
            a_{31} & a_{32} \\
        \end{vmatrix}
\]

Znaménka u jednotlivých prvků plynou ze známének permutací ve výpočtu determinantu,
můžeme však použít pomůcku, která říka, že pokud součet indexů prvku pro který počítáme
minor bude sudý, bude mít znaménko $+$ a pokud bude součet lichý, bude mít znaménko $-$.

Postup, který byl naznačen na matici řádu $3$ funguje obecně pro libovolné matice řádu $n$.

\begin{example}[Výpočet determinantu pomocí Laplaceova rozvoje]
    \[
      d = \begin{vmatrix}
          2 & 1 & 5 & -1 & \tikzmark{l1}0 \\
          3 & 2 & 4 & 0 & 0  \\
          1 & 0 & -2 & 1 & 1 \\
          3 & 2 & 0 & 5 & 0  \\
          2 & 0 & -2 & 4 &  9 \tikzmark{r1}
      \end{vmatrix}
      \DrawBox[dashed, red, rounded corners=0.5ex ]{l1}{r1}{\textcolor{red}{}}
      % \DrawBox[thick, red, rounded corners=2ex ]{left1}{right1}{\textcolor{red}{\footnotesize$s^1$}}
    % \DrawBox[thick, blue, dashed]{left2}{right2}{\textcolor{blue}{\footnotesize$s^2$}}
    \]

    \[ d = 1 \cdot
    \begin{vmatrix}
        2 & 1 & 5 & -1 \\
        \tikzmark{l1} 3  & 2 & 4 & 0 \tikzmark{r1}\\
        3 & 2 & 0 & 5 \\
        2 & 0 & -2 & 4
        \DrawBox[dashed, red, rounded corners=0.5ex ]{l1}{r1}{\textcolor{red}{}}
    \end{vmatrix}
     + 9 \cdot
    \begin{vmatrix}
        2 & 1 & 5 & -1 \\
        \tikzmark{l1} 3  & 2 & 4 & 0 \tikzmark{r1}\\
        1 & 0 & -2 & 1 \\
        3 & 2 & 0 & 5
        \DrawBox[dashed, red, rounded corners=0.5ex ]{l1}{r1}{\textcolor{red}{}}
    \end{vmatrix}
    \]

    \[d =
        -3 \cdot
        \begin{vmatrix}
            1 & 5 & -1 \\
            2 & 0 & 5 \\
            0 & -2 & 4
        \end{vmatrix}
        + 2 \cdot
        \begin{vmatrix}
            2 & 5 & -1 \\
            3 & 0 & 5 \\
            2 & -2 & 4
        \end{vmatrix}
        - 4 \cdot
        \begin{vmatrix}
         2 & 1 & -1 \\
         3 & 2 & 5 \\
         2 & 0 & 4
        \end{vmatrix} + 9 \cdot \Bigg (
        -3 \cdot
        \begin{vmatrix}
            1 & 5 & -1 \\
            0 & -2 & 1 \\
            2 & 0 & 5
        \end{vmatrix}
        + 2 \cdot
        \begin{vmatrix}
            2 & 5 & -1 \\
            1 & -2 & 1 \\
            3 & 0 & 5 \\
        \end{vmatrix}
        - 4 \cdot
        \begin{vmatrix}
            2 & 1 & -1 \\
            1 & 0 & 1 \\
            3 & 2 & 5
        \end{vmatrix} \Bigg)
    \]
\end{example}

S použitím této metody je možné také vybrat více řádků/sloupců současně, což v některých
případech může být výhodné.

Mějme matici $A$ řádu $n$ a indexy $$i, j = 1, \ldots n$$
Potom vybereme řádky $$i_1 \ldots i_q, 1 \leq q \leq n$$
Potom:
$$det(A) =(-1)^{i_1 + \ldots i_q + 1 + \ldots + q}
A_{i_1, \ldots, i_q; 1,\ldots, q} M_{i_1, \ldots, i_q; 1,\ldots, q} + \ldots +
(-1)^{i_1, \ldots, i_q; n - q + 1,\ldots, n} \cdot
A_{i_1, \ldots, i_q; n - q + 1,\ldots, n} M_{i_1, \ldots, i_q; n - q + 1,\ldots, n}
$$

\begin{example}[Laplaceova metoda s výběrem více řádků/sloupců]
\[d =
    \begin{vmatrix}
        2 & 1 & 0 & 1 & 5 \\
        \tikzmark{l1}0 & 0 & 2 & 0 & 7 \tikzmark{r1}\\
        3 & 1 & 5 & 2 & -2 \\
        \tikzmark{l2} 1 & 0 & 0 & 0 & 0 \tikzmark{r2}\\
        2 & 5 & 2 & 3 & -8
    \end{vmatrix}
    \DrawBox[dashed, red, rounded corners=0.5ex ]{l1}{r1}{\textcolor{red}{}}
    \DrawBox[dashed, red, rounded corners=0.5ex ]{l2}{r2}{\textcolor{red}{}}
\]

\[d =
(-1)^{2 + 4 + 1 + 2} \cdot
\begin{vmatrix}
    0 & 0\\
    1 & 0
\end{vmatrix}
\cdot
\begin{vmatrix}
    0 & 1 & 5\\
    5 & 2 & -2\\
    2 & 3 & -8
\end{vmatrix}
\ldots
\]
Takových členů bude obecně $\binom{n}{q}$, kde $q$ označuje počet vybraných řádků/sloupců
v tomto případě tedy $\binom{5}{2}$. Díky vhodně vybraným řádkům pro vytvoření rozvoje však velká
část těchto členů bude ve výsledku nulová a ty můžeme tedy stejně jako v předchozím příkladu rovnou
vynechat a psát pouze ty nenulové a stejně tak můžeme rovnou vyjádřit hodnotu subdeterminantu,
pokud je zřejmá.
\[d =
-2 \cdot
\begin{vmatrix}
    1 & 1 & 5 \\
    1 & 2 & -2 \\
    5 & 3 & -8
\end{vmatrix}
-7 \cdot
\begin{vmatrix}
    1 & 0 & 1 \\
    1 & 5 & 2 \\
    5 & 2 & 3 \\
\end{vmatrix}
\]
\end{example}

\subsection{Inverzní matice}
\begin{definition}[Algebraický doplňek]
    Algebraický doplněk prvku $a_{ij}$ je $(-1)^{i+j}\cdot M_{ij}$.
\end{definition}

\begin{definition}[Adjungovaná matice]
    Adjungovaná matice $A^*$ vznikne z matice $A$ tak, že každý prvek
    nahradíme jeho algebraickým doplňkem.
\end{definition}

\begin{definition}[Inverzní matice]
    Inverzní matice, je matice k dané matici A, která splňuje:

    $$A \cdot A^{-1} = E = A^{-1} \cdot A$$

    Inverzní matici můžeme spočítat jako:
    $$A^{-1} = \frac{1}{|A|}\cdot (A^*)^{T}$$

    Z Cauchyho věty o součinu \ref{the:cauchy} vyplývá, že determinant A musí být nenulový\footnote{Matice s nenulovým determinantem
    nazýváme regulární, a naopak matice s nulovým determinantem nazýváme singulární.}.
\end{definition}

\begin{example}[Výpočet inverzní matice]
    Spočítejte inverzní matici $A^{-1}$ k matici $A$:
    \[A =
        \begin{pmatrix}
            1 & 2 & 0\\
            2 & -1 & 3 \\
            3 & 0 & 4 \\
        \end{pmatrix}
    \]
    Prvním krokem je zjistit, zda je $A$ vůbec regulární matice, tedy spočítat determinant $A$.
    $$|A| = 1 \cdot (-1) \cdot 4 + 3 \cdot 2 \cdot 3 - (4 \cdot 2 \cdot 2) = -2$$
    Následně potřebujeme vyjádřit adjungovanou matici $A^+$
    \[A^*=
        \begin{pmatrix}
            \begin{vmatrix}
                -1 & 3 \\
                0 & 4
            \end{vmatrix} &
            - \begin{vmatrix}
                2 & 3 \\
                3 & 4
            \end{vmatrix} &
            \begin{vmatrix}
                2 & -1 \\
                3 & 0
            \end{vmatrix} \\

            - \begin{vmatrix}
                2 & 0 \\
                0 & 4
            \end{vmatrix} &
            \begin{vmatrix}
                1 & 0 \\
                3 & 4
            \end{vmatrix} &
            - \begin{vmatrix}
                1 & 2 \\
                3 & 0
            \end{vmatrix} \\

            \begin{vmatrix}
                2 & 0 \\
                -1 & 3
            \end{vmatrix} &
            - \begin{vmatrix}
                1 & 0 \\
                2 & 3
            \end{vmatrix} &
            \begin{vmatrix}
                1 & 2 \\
                2 & -1
            \end{vmatrix}
        \end{pmatrix}
        = \begin{pmatrix}
            -4 & 1 & 3 \\
            -8 & 4 & 6 \\
            6 & -3 & -5
        \end{pmatrix}
    \]
    \[
    A^{-1} = \frac{1}{|A|} \cdot {A^*}^T = \frac{1}{-2} \cdot
        \begin{pmatrix}
            -4 & -8 & 6\\
            1 & 4 & -3 \\
            3 & 6 & -5
        \end{pmatrix}
        = \begin{pmatrix}
            2  & 4 & -3 \\
            \frac{-1}{2}  & -2 & \frac{3}{2} \\
            \frac{-3}{2} & -3 & \frac{5}{2}
        \end{pmatrix}
    \]
    Pro ověření:
    \[
      A \cdot A^{-1} =
      \begin{pmatrix}
          1 & 0 & 0\\
          0 & 1 & 0\\
          0 & 0 & 1
      \end{pmatrix}
    \]
 \end{example}

 Pro každou regulární matici jsme schopni tímto způsobem najít matici inverzní.
 Vezmeme li množinu regulárních matic řádu n a operaci násobení matic, zjistíme,
 že toto násobení je asociativní, ke každé matici existuje matice inverzní a vzhledem
 k násobení máme i neutrální prvek, kterým je jednotková matice. Regulární matice nad
 polem $F$ řádu $n$ s operací násobení matic tvoří grupu! Tuto grupu značíme jako:
 $$GL(n, F)$$

\begin{example}[Počet prvků grupy GL nad konečným polem]
    Kolik prvků má $GL(n, \mathbb{F}_{2^k})$?
\end{example}

\begin{example}[Nad polem $F_7$ řešte maticovou rovnici]
    $$X\cdot A = B$$
    Pro neznámou matici $X$ a
    \[
    A = \begin{pmatrix}
        2 & 3 \\
        5 & 1
    \end{pmatrix}, \;
    B = \begin{pmatrix}
        3 & 0 \\
        5 & 6
    \end{pmatrix}
    \]

Je nutné pamatovat na to, že násobení matic není komutativní a správně vynásobit obě
strany korektně inverzní maticí.
    \begin{align*}
        X \cdot A &= B \\
        X \cdot A \cdot A^{-1} &= B \cdot A^{-1}\\
        X &= B \cdot A^{-1}
    \end{align*}
Vyjádříme inverzní matici $A^{-1}$:
\[A^{-1} =\frac{1}{|A|} \cdot {A^*}^T = \frac{1}{1} \cdot
\begin{pmatrix}
    1 & 2 \\
    4 & 2
\end{pmatrix}^T = \begin{pmatrix}
    1 & 4 \\
    2 & 2
\end{pmatrix}
\]
A dosadíme do dříve získaného vztahu:
    \begin{align*}
        X &=
        \begin{pmatrix}
            3 & 0 \\
            5 & 6
        \end{pmatrix} \cdot
        \begin{pmatrix}
            1 & 4 \\
            2 & 2
        \end{pmatrix} \\
        X &=
        \begin{pmatrix}
        3 & 5 \\
        3 & 4
        \end{pmatrix}
    \end{align*}
\end{example}

\subsection{Hodnost matice}
Z definice \ref{def:lin_nezavislost} víme co je lineární nezávislost vektorů. V každé
matici můžeme řádky (případně sloupce) uvažovat jako vektory. Může nás zajímat, kolik je v
matici lineárně nezávislých řádků (případně sloupců).

Počet lineárně nezávislých řádků se nazývá hodnost matice. Hodnost matice je tedy počet jejich
lineárně nezávislých řádků, což je totéž jako počet lineárně nezávislých sloupců. Hodnost
matice $A$ značíme:
$$h(A)$$

Nejmenší hodnost matice může být 0, hodnost 0 má pouze nulová matice, protože nulový vektor
není nezávislý ani sám o sobě. A maximální možná hodnost pro matici o rozměrech
$m \times n$ je $min(m, n)$.
$$0 \leq h(A) \leq min(m, n)$$


\subsubsection{Výpočet hodnosti matice}
Pro jednotlivé řádky by bylo možné využít vztahu z definice \ref{def:lin_nezavislost},
pomocí něj sestavit soustavu lineárních rovnic a vypočítat její řešení. To však může být
zbytečně zdlouhavé, budeme tedy používat tzv. ekvivalentní (elementární) úpravy matice,
které nemění hodnost matice.

Ekvivalentními úpravami matice jsou:
\begin{enumerate}
    \label{ekv_upravy}
    \item Přičtení k-násobku řádku k jinému řádku.
    \item Vynásobení řádku nějakým nenulovým číslem $l$.
    \item Výměna řádků\footnote{Výměna řádků jde relizovat pomocí prvních dvou úprav,
    její uvádění zde tedy není nutné.}.
\end{enumerate}

Pomocí těchto ekvivalentních úprav upravíme matici na schodovitou. A hodnost schodovité
matice je velice snadné spočítat, protože u schodovité matice je její hodnost počet nenulových
řádků (vektorů).

Maximální hodnost matice označujeme jako plnou hodnost.

\begin{example}[Výpočet hodnosti matice]
    Vypočtěte hodnost matice $Q$.
    \[Q =
    \begin{pmatrix}
        1 & 2 & 0 & 5 \\
        2 & -1 & 2 & -3\\
        2 & 1 & -3 & 6\\
        6 & -1 & 1 & 0
    \end{pmatrix}
    \]

    Pomocí elementárních úprav převedeme matici $Q$ do schodovítého tvaru.
    \begin{align*}
        Q \sim
        \begin{pmatrix}
            1 & 2 & 0 & 5 \\
            0 & -5 & 2 & -13\\
            0 & -3 & -3 & -4\\
            0 & -13 & 1 & -30
        \end{pmatrix} \sim
        \begin{pmatrix}
            1 & 2 & 0 & 5 \\
            0 & -5 & 2 & -13\\
            0 & 15 & 15 & 20\\
            0 & 65 & -5 & 150
        \end{pmatrix} \sim
        \begin{pmatrix}
            1 & 2 & 0 & 5 \\
            0 & -5 & 2 & -13\\
            0 & 0 & 21 & -19\\
            0 & 0 & 21 & -19
        \end{pmatrix} \sim
        \begin{pmatrix}
            1 & 2 & 0 & 5 \\
            0 & -5 & 2 & -13\\
            0 & 0 & 21 & -19\\
            0 & 0 & 0 & 0
        \end{pmatrix} = R
    \end{align*}

    Z upravené matice vidímě, že:
    $$h(Q) = h(R) = 3$$
\end{example}

\subsection{Výpočet determinantu pomocí ekvivalentních úprav}
S využitím ekvivalentních úprav \ref{ekv_upravy} můžeme matici převést na
schodovitý tvar a jednoduše spočítat determinant schodovité matice jako
součin prvků na hlavní diagonále.

Při aplikaci úprav však musíme dávat pozor na to, jakým způsobem která úprava mění
determinant upravené matice:

\begin{enumerate}
    \item Přičtení k-násobku řádku k jinému řádku. \hfill Determinant se nemění.
    \item Vynásobení řádku nějakým nenulovým číslem $l$. \hfill Determinant bude $l$ krát větší.
    \item Výměna řádků.\hfill Změní se znaménko determinantu.
\end{enumerate}

Vliv těchto úprav na determinant je lehké ukázat na matici řádu 2. Platí však obecně pro
jakoukoliv matici řádu $n$.

Z těchto pravidel lze také vyvodit, že každá čtvercová matice řádu $n$ s plnou hodností
je regulární a s jakoukoliv menší než plnou hodností je singulární.

\begin{example}[Výpočet determinantu pomocí ekvivalentních úprav]
    Pomocí ekvivalentních úprav spočtěte determinant matice M.
    \[M=
    \begin{pmatrix}
        1 & 2 & 3 & 4 \\
        2 & 4 & 6 & 2 \\
        0 & 0 & 5 & 6 \\
        0 & 1 & 10 & 1
    \end{pmatrix}
    \eqop{r_2 \cdot  \frac{1}{2}}
    \begin{pmatrix}
        1 & 2 & 3 & 4 \\
        1 & 2 & 3 & 1 \\
        0 & 0 & 5 & 6 \\
        0 & 1 & 10 & 1
    \end{pmatrix} \eqop{r_2 - r_1}
    \begin{pmatrix}
        1 & 2 & 3 & 4 \\
        0 & 0 & 0 & -3 \\
        0 & 0 & 5 & 6 \\
        0 & 1 & 10 & 1
    \end{pmatrix} \eqop{r_2 \leftrightarrow r_4}
    \begin{pmatrix}
        1 & 2 & 3 & 4 \\
        0 & 1 & 10 & 1 \\
        0 & 0 & 5 & 6 \\
        0 & 0 & 0 & -3
    \end{pmatrix} = N
    \]

    \begin{align*}
        det(N) &= -15\\
        det(N) &= \frac{1}{2} \cdot -1 \cdot det(M)\\
        det(M) &= -2 \cdot det(N) = 30
    \end{align*}
\end{example}

\begin{definition}[Symetrická matice]
    Matice $A$ je symetrická, pokud platí:
    $$A^T = A$$
    Pro jednotlivé prvky matice musí tedy platit:
    $$a_{ij} = a_{ji}$$
    Symetrickou matici značíme s dolním indexem $sym$, např. $A_{sym}$
\end{definition}

\begin{definition}[Antisymetrická matice]
    Matice $A$ je antisymetrická, pokud platí:
    $$A^T = -A$$
    Pro jednotlivé prvky matice musí tedy platit:
    $$a_{ij} = -a_{ji}$$
    Antisymetrickou matici značíme s dolním indexem $alt$, např. $A_{alt}$
\end{definition}

\begin{theorem}[Rozklad čtvercové matice na symetrickou a antisymetrickou matici]
    Každou čtvercovou matici lze rozložit na součet matice symetrické a antisymetrické.

    $$A = A_{sym} + A_{alt}$$
\end{theorem}
\begin{proof}
    Matice $A_{sym}$ a $A_{alt}$ můžeme vždy zkonstruovat následovně:
    $$A_{sym} = \frac{1}{2} \cdot (A + A^T)$$
    Takto zvolená matice $A_{sym}$ je určitě symetrická, protože:
    $$(A_{sym})^T = \frac{1}{2}(A^T + (A^T)^T) = A_{sym}$$

    Podobně pro $A_{alt}$:
    $$A_{alt} = \frac{1}{2}\cdot (A - A^T)$$
    Takto zvolená matice $A_{alt}$ je určitě antisymetrická, protože:
    $$(A_{alt})^T = \frac{1}{2} \cdot (A^T - (A^T)^T) = - A_{alt}$$

    Zároveň vidíme, že výsledkem součtu těchto dvou matic je opravdu původní matice:
    $$A_{sym} + A_{alt} = \frac{1}{2} \cdot (A + A^T) + \frac{1}{2}\cdot (A - A^T) = A$$
\end{proof}

\subsection{Soustavy linearních rovnic}
Soustavy lineárních rovnic můžeme řešit elementárně (vyjadřovat neznámé a dosazovat
do ostatních rovnic, případně sečíst dvě rovnice a tím nějakou neznámou eliminovat). Tento
způsob však má nevýhodu, že není algoritmický.

Proto budeme používat postupy pomocí matic, které jsou jednoduše algoritmizovatelné.

Soustava lineárních rovnic obecně:
\begin{align*}
    a_{11} \cdot x_1 + \ldots a_{1n} \cdot x_n &= b_1\\
    \vdots\\
    a_{m1} \cdot x_1 + \ldots a_{mn} \cdot x_n &= b_m\\
\end{align*}
Proměnným $a_{ij}$ říkáme koeficienty, které tvoří obdélníkovou matici
o $m$ řádcích a $n$ sloupcích. Proměnné $x_1, \ldots, x_n$ nazýváme neznámé, ty budeme
počítat. A proměnné $b_1, \ldots, b_m$ nazýváme absolutní členy.
Tuto obecnou soustavu lineárních rovnic můžeme zapsat maticově:
\[
    \begin{pmatrix}
        a_{11} & \ldots & a_{1n} \\
        \vdots & \vdots & \vdots\\
        a_{m1} & \ldots & a_{mn}
    \end{pmatrix}
    \begin{pmatrix}
        x_1 \\
        \vdots \\
        x_m
    \end{pmatrix}
    =
    \begin{pmatrix}
        b_1 \\
        \vdots \\
        b_m
    \end{pmatrix}
\]

Někdy také zapisujeme jako:
$$A \cdot \vec{x} = \vec{b}$$

V případě, že je vektor absolutních členů $\vec{b}$ nulový vektor, tedy:
$$\vec{b} = \vec{o}$$
Nazýváme takovou soustavu homogenní. Homogenní soustava lineárních rovnic je vždy řešitelná.
Obecně řešení homogenní soustavy lineárních rovnic tvoří vektorový podprostor.

V případě nehomogenních soustav se nejedná o vektorový podprostor, ale o affinní podprostor.


\subsubsection{Gaussova eliminační metoda}
Matici soustavy\footnote{Tuto matici tvoří jednotlivé koeficienty.} $A$ a vektor absolutních členů
$\vec{b}$ zapíšeme do jedné rozšířené matice následovně:
\begin{align*}
    \begin{pmatrix}[c|c]
        A & \vec{b}
    \end{pmatrix} =
    \begin{pmatrix}[ccc|c]
        a_{11} & \ldots & a_{1n} & b_1\\
        \vdots & \vdots & \vdots & \vdots \\
        a_{m1} & \ldots & a_{mn} & b_m
    \end{pmatrix}
\end{align*}
Tuto rozšířenou matici poté pomocí elementárních úprav upravujeme do vhodného (schodovitého) tvaru,
ze kterého dokážeme lehce zjistit řešení celé soustavy. Ze schodovité matice pak jednoduše dostaneme
řešení celé soustavy "zpětným chodem" od posledního řádku, kdy postupně zjišťujeme hodnoty neznámých.

Soustava rovnice je řešitelná právě tehdy, když:
\[
    h(A) = h\Bigg (
    \begin{pmatrix}[c|c]
        A & \vec{b}
    \end{pmatrix} \Bigg )
\]
V opačném případě matice řešitelná není.

V případě, že se hodnosti rovnají, a soustava je tedy řešitelná, mohou nastat dva případy:
\begin{enumerate}
    \item $h(A) = n$, kde n je počet neznámých \hfill Rovnice má právě jedno řešení.
    \item $h(A) = k < n$ \hfill Rovnice má nekonečně mnoho řešení\footnote{A tato řešení jsou
    závislá na $n-k$ libovolných parametrech. V případě, že navíc pracujeme nad konečným polem, není jich
    tedy nekonečně mnoho}.
\end{enumerate}
Mohou tedy nastat pouze 3 případy:
\begin{enumerate}
    \item Soustava není řešitelná.
    \item Soustava je řešitelná a má právě jedno řešení.
    \item Soustava je řešitelná a má nekončně mnoho řešení.
\end{enumerate}

\begin{example}[Příklad neřešitelné soustavy]
    Řešte následujicí soustavu rovnic:
    \begin{align*}
        x + y + z &= 2 \\
        x - y + 2z &= 1 \\
        7x + 4y + 7z &= -3
    \end{align*}

    Tuto soustava přepíšeme do matice a upravíme do schodovitého tvaru:
    \[
        \begin{pmatrix}[ccc|c]
            1 & 1 & 1 & 2 \\
            2 & -1 & 2 & 1 \\
            7 & 4 & 7 & -3
        \end{pmatrix} \eqop{r_2 - 2\cdot r_1, r_3 - 7\cdot r_1}
        \begin{pmatrix}[ccc|c]
            1 & 1 & 1 & 2 \\
            0 & -3 & 0 & -3 \\
            0 & -3 & 0 & -17
        \end{pmatrix} \eqop{r_3 - r_2}
        \begin{pmatrix}[ccc|c]
            1 & 1 & 1 & 2 \\
            0 & -3 & 0 & -3 \\
            0 & 0 & 0 & -14
        \end{pmatrix}
    \]
    Hodnost základní matice je v tomto případě zjevně 2 a hodnost
    celé rozšířené matice je zjevně 3. Hodnosti se nerovnají a soustava tedy nemá řešení.
\end{example}
